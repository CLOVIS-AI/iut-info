\documentclass[10pt,a4paper,french]{article}
\usepackage{babel}
\usepackage[utf8]{inputenc}
\usepackage[T1]{fontenc}
\usepackage{../latex/clovisai}

\begin{document}

\title{Réseau}
\author{Ivan Canet}
\maketitle

\begin{abstract}
Ceci est l'introduction du document.
\end{abstract}
\tableofcontents

\section{Configuration réseau via IP}

\subsection{Adresse IP}

Une adresse IPv4 est constituée de 32 bits, elle est utilisée pour communiquer entre machines sur le réseau.

Dans un réseau local, on ``bloque'' le début de l'adresse et on utilise que la fin, par exemple l'adresse de réseau `221.164.0.0/24' bloque les 24 premiers bits, ce qui veut dire que les adresses de ce réseau vont de `221.164.0.1' à `221.164.0.254'.

Dans un réseau, l'adresse finissant par un 0 est l'adresse du réseau (l'adresse de la box/routeur) et l'adresse finissant par 255 est l'adresse de broadcast, qui permet d'envoyer un paquet à tout le monde.

Le réseau `127.0.0.0/8' est un faux réseau qui revient sur soi-même, il est pratique pour tester les applications en réseau, etc. Toutes les adresses commençant par 127 ne partent donc pas sur le réseau, souvent on utilise `127.0.0.1'.

Chaque machine peut se connecter à plusieurs `interfaces' : elles correspondent aux différentes prises sur la carte réseau. Par exemple, la première prise ethernet est nommée `eth0', puis `eth1', etc.

\subsection{Commandes utiles}

On utilise {\tt dev} pour ``device''.

\paragraph{Afficher adresses IP et interfaces}

\begin{minted}{bash}
ip a
ifconfig
\end{minted}

\paragraph{Afficher les interfaces activées}

\begin{minted}{bash}
ip link ls up
\end{minted}

\paragraph{Assigner une adresse IP et un masque à une interface}

\begin{minted}{bash}
ip a add 192.168.74.101/24 dev eth0
ifconfig eth0 192.168.74.101 netmask 255.255.255.0
\end{minted}

\paragraph{Activer le mode routeur}

\begin{minted}{bash}
sysctl -w net.ipv4.ip\_forward=1
\end{minted}

\paragraph{Voir la table de routage}

\begin{minted}{bash}
ip r
\end{minted}

\paragraph{Ajouter une route par défaut}

\begin{minted}{bash}
ip r add default via 192.168.74.254
route add -net default gw 192.168.74.254
\end{minted}

\paragraph{Ajouter une route}

\begin{minted}{bash}
ip r add 22.33.44.0/24 via 11.22.33.55
route add -net 22.33.44.0 netmask 255.255.255.0 gw 11.22.33.55
\end{minted}

\paragraph{Activer/Désactiver une interface}

\begin{minted}{bash}
ip link set dev eth0 up|down
ifup|ifdown eth0
\end{minted}

\subsection{Configurer de manière pérenne}

Il faut modifier le fichier {\tt etc/network/interfaces}.

\subsubsection{Pour un client}

\begin{minted}{bash}
auto eth0
iface eth0 inet static
	address 10.1.1.22                    # IP de la machine
	netmask 255.255.255.0                # Masque de sous-réseau
	gateway 10.1.1.254                   # Route par défaut
\end{minted}

\subsubsection{Pour un routeur}

\begin{minted}{bash}
auto eth0
iface eth0 inet static
	address 10.1.1.254                   # IP de la machine
	netmask 255.255.255.0                # Masque de sous-réseau
	up sysctl -w net.ipv4.ip_forward=1   # Activer le port forwarding
\end{minted}

On utilise {\tt up} pour lancer une commande au démarrage de l'interface (ici, l'activation du port forwarding, mais on peut mettre n'importe quelle commande).

\section{Configuration réseau via DHCP}

\subsection{Protocole DHCP}

Le protocole DHCP permet de reléguer la configuration. Au lieu d'enregistrer la configuration dans le client, on va créer un serveur DHCP et le client va lui demander son IP.

\subsubsection{Trames}

\paragraph{Initialisation}

Le client demande en broadcast une adresse IP au serveur, le serveur lui en propose une, le client dit qu'il l'accepte (toujours en broadcast), et enfin le serveur confirme.

Ces étapes sont importantes dans le cas où il y a plusieurs serveurs DHCP : le client ``choisit'' quelle offre il souhaite, et on évite qu'un client ai 2 IPs.

\begin{center}
\begin{sequencediagram}
	\newthread{client}{Client}
	\newthread{server}{Serveur DHCP}
	
	\mess[1]{client}{DISCOVER}{server}
	\mess[1]{server}{OFFER}{client}
	\mess[1]{client}{REQUEST}{server}
	\mess[1]{server}{ACK}{client}
\end{sequencediagram}
\end{center}

\paragraph{Libération}

Le client peut annoncer qu'il quitte le réseau, son adresse IP est alors libérée. Ce n'est absolument pas obligatoire, c'est plus de la politesse.

\begin{center}
\begin{sequencediagram}
	\newthread{client}{Client}
	\newthread{server}{Serveur DHCP}
	
	\mess[1]{client}{RELEASE}{server}
\end{sequencediagram}
\end{center}

\paragraph{Renouvellement à mi-bail}

À la moitié du bail, le client fait une demande de renouvellement. Cette fois, la demande est faite en direct au serveur, pas en broadcast.

\begin{center}
\begin{sequencediagram}
	\newthread{client}{Client}
	\newthread{server}{Serveur DHCP}
	
	\mess[1]{client}{REQUEST}{server}
	\mess[1]{server}{ACK}{client}
\end{sequencediagram}
\end{center}

\paragraph{Renouvellement en fin de bail}

Si le serveur n'a pas répondu lors du mi-bail, le client fait une nouvelle demande en broadcast à 87.5\% du bail (elle fait une demande à tous les serveurs DHCP, pour le cas où le précédent a un problème).

\begin{center}
\begin{sequencediagram}
	\newthread{client}{Client}
	\newthread{server}{Serveur DHCP}
	
	\mess[1]{client}{REQUEST}{server}
	\mess[1]{server}{ACK}{client}
\end{sequencediagram}
\end{center}

\subsubsection{Messages du protocole}

\paragraph{Envoyés par le client}

\begin{description}
\item[DISCOVER] Qui sont les serveurs DHCP disponibles ?
\item[REQUEST] Est-ce que je peux avoir un bail ?
\item[DECLINE] Cette adresse ne me plaît pas.
\item[RELEASE] Je n'ai plus besoin de l'adresse, tu peux la réutiliser.
\item[INFORM] Je veux plus d'informations sur le réseau (liste des imprimantes...)
\end{description}

\paragraph{Envoyés par le serveur}

\begin{description}
\item[OFFER] Voilà une IP, est-ce que tu la veux?
\item[ACK] C'est bon, le bail est validé.
\item[NACK] Je refuse de donner un bail.
\end{description}

\subsection{Configuration}

\subsubsection{Côté client}

L'énorme avantage d'utiliser DHCP est que la configuration côté client est très faible.

On modifie le fichier {\tt /etc/network/interfaces}:

\begin{minted}{bash}
auto eth0
iface eth0 inet dhcp
\end{minted}

La liste des baux accordés est disponible dans {\tt /var/lib/dhcp/dhclient.eth0.leases} (remplacer {\tt eth0} par la bonne interface).

Le port 68 est utilisé par le client.

\subsubsection{Côté serveur}

On modifie le fichier {\tt /etc/dhcp/dhcpd.conf}:

\begin{minted}{bash}
# Temps par défaut pour les baux, en secondes
default-lease-time 600;
# Durée maximale d'un bail, en secondes
max-lease-time 7200;

# Le réseau a pour IP 192.168.10.0/24
# Attention, ce n'est pas l'adresse du serveur !
subnet 192.168.10.0 netmask 255.255.255.0 {
  # Les clients auront une IP comprise entre 192.168.10.100 et 192.168.10.150
  # À noter que le serveur DHCP lui-même ne doit pas être dans cette range!
  range 192.168.10.100 192.168.10.150;
  # On dit que la machine a0:00:01:02:03:04 a une adresse IP réservée : 192.168.10.200
  # On nomme la machine `server' ; le nom n'a pas d'importance
  # À noter que l'IP réservée N'EST PAS dans la range précisée plus haut!
  host server {
    hardware-ethernet a0:00:01:02:03:04;
    fixed-address 192.168.10.200;
  }
  # Les clients utiliseront le routeur 192.168.10.1 comme route par défaut (gateway)
  option routers 192.168.10.1;
}
\end{minted}

Il faut penser à relancer le serveur à chaque fois que l'on modifie sa configuration:
\begin{minted}{bash}
service isc-dhcp-server restart   # Relancer le serveur
cat /var/lib/dhcp/dhcpd.leases    # Afficher les baux
> /var/lib/dhcp/dhcpd.leases      # Supprimer les baux
cat /var/log/syslog | tail        # Afficher les logs
\end{minted}

Le port 67 est utilisé par le serveur.

\section{DNS}

\subsection{Définition}

Un serveur DNS permet de relier des noms de domaines (eg. {\tt clovis.online}) à leur adresse IP. Il existe des serveurs `maîtres' (les originaux) et les serveurs `esclaves' (qui copient les maîtres).

La racine (appelée `.') est composée de 13 serveurs maîtres et plus de 130 esclaves.

On peut utiliser la commande {\tt host} pour interroger les serveurs DNS pour obtenir une adresse IP:
\begin{minted}{bash}
> host www.google.fr
www.google.fr has address 172.217.19.131
www.google.fr has IPv6 address 2a00:1450:4006:806::2003
\end{minted}

Les serveurs DNS gèrent aussi les mails.

\paragraph{Records}
Les différentes données des serveurs DNS sont classées en catégories, nommées `records':
\begin{description}
\item[A] Adresses IPv4
\item[AAAA] Adresses IPv6
\item[MX] Serveurs de mail
\item[CNAME] Alias (rediriger vers un autre nom de domaine)
\end{description}

On peut aussi utiliser {\tt host} dans l'autre sens, pour trouver à quel nom de domaine correspond une adresse IP:
\begin{minted}{bash}
> host 172.217.19.131
131.19.217.172.in-addr.arpa domain name pointer mrs08s04-in-f3.1e100.net.
131.19.217.172.in-addr.arpa domain name pointer par03s12-in-f131.1e100.net.
\end{minted}

\subsection{Configuration côté client}

\paragraph{/etc/resolv.conf}
\begin{minted}{bash}
search [DOMAINE]
domain [DOMAINE]
nameserver [IP]
\end{minted}
Cela permet de donner l'IP du DNS du domaine. La valeur donnée au {\tt search} permet de donner un serveur par défaut : si on donne une adresse qui n'est pas complète (eg. {\tt ping toto}), le client ajoute ce qu'on a donné à {\tt search}.

\paragraph{/etc/nsswitch.conf}
\begin{minted}{bash}
hosts:	files	dns
\end{minted}

\subsection{Configuration côté serveur}

\subsubsection{Déclaration des zones}

\paragraph{/etc/bind/named.conf.local}
\begin{minted}{bash}
zone "[DOMAINE]" {
    type [TYPE];
    file "/etc/bind/db-[DOMAINE].net";
};

zone "[IP].in-addr.arpa" {
    type [TYPE];
    file "/etc/bind/db-[IP]";
};
\end{minted}
où l'on remplace les valeurs par:
\begin{description}
\item[\tt DOMAINE] est le nom du domaine duquel ce serveur est responsable,
\item[\tt TYPE] vaut `master' pour un serveur maître,
\item[\tt IP] est l'IP du sous-domaine (sans le cache) à l'envers (eg. pour 152.168.1.0/24, on utilise 1.168.152).
\end{description}

\subsubsection{Création d'une zone}

\paragraph{/etc/bind/db-[{\tt DOMAINE}].net}
On peut utiliser un `;' dans ce fichier pour faire des commentaires.
\begin{minted}{bash}
@ IN SOA [DOMAINE] [MAIL] (
    [SERIAL]
    [REFRESH]
    [RETRY]
    [EXPIRE]
    [NEGATIVE CACHE TTL]
)

@              IN NS     [DNS]
@              IN MX [N] [SERVEUR MAIL]
[SOUS-DOMAINE] IN A      [IP SOUS-DOMAINE]
[ALIAS]        IN CNAME  [DIRECTION]
[SOUS-DNS]     IN NS     [IP SOUS-DNS]
\end{minted}
explications:
\begin{itemize}
\item {\tt @} représente le nom de domaine (remplacé automatiquement)
\item {\tt DOMAINE} est le nom du domaine géré par ce serveur. On peut soit préciser le nom complet du domaine avec un . à la fin (eg. {\tt google.fr.}) ou le nom du sous-domaine, qui sera ajouté automatiquement au domaine parent ({\tt google}).
\item {\tt MAIL} est l'adresse mail de l'administrateur du serveur DNS. Elle est notée avec un . à la place de l'arobase (eg. {\tt contact@clovis.online} devient {\tt contact.clovis.online}).
\item {\tt SERIAL} est le numéro de version (souvent on utilise la date puis deux chiffres pour la version du jour ; eg. 2018031601 pour la première version du 16 mars 2018).
\item {\tt REFRESH} est le temps entre deux remises à jour des esclaves (secondes).
\item {\tt RETRY} est le temps entre deux connections des esclaves si le maître n'est pas disponible (secondes).
\item {\tt EXPIRE} est le temps à partir duquel les esclaves considèrent le maître comme inexistant, s'ils n'arrivent pas à l'atteindre (secondes).
\item {\tt NEGATIVE CACHE TTL} est le temps pendant lequel les informations de redirections sont gardées en mémoire (au cas où la même demande est faite dans pas longtemps) (secondes).
\item {\tt DNS} est le nom du domaine (par exemple google.fr.).
\item {\tt N} est le numéro de priorité de ce serveur mail (de 1 à 10).
\item {\tt SERVEUR MAIL} est le serveur qui gère les mails (par exemple `gmail' qui sera remplacé par `gmail.google.fr.').
\item {\tt SOUS-DOMAINE} un sous-domaine dont va préciser l'IP;
\item {\tt IP SOUS-DOMAINE} est l'IP de ce sous-domaine.
\item {\tt ALIAS} permet de créer un sous-domaine qui est une redirection sur un autre domaine.
\item {\tt DIRECTION} est un autre nom de domaine sur lequel on redirige.
\item {\tt SOUS-DNS} est un sous-domaine qui a son propre DNS.
\item {\tt IP SOUS-DNS} est l'IP du sous-domaine avec un DNS.
\end{itemize}

Ne pas oublier d'augmenter le numéro de version à chaque modification ({\tt SERIAL}) !

Exemple de fichier (pour le domaine `entree.maison.net'):
\begin{minted}{bash}
@ IN SOA entree.maison.net. admin.maison.net. (
    2018111601
    604800
    86400
    2419200
    604800
)

@              IN NS     entree.maison.net.
@              IN MX 10  entree
entree         IN A      10.4.2.254
vestiaire      IN CNAME  entree
etage          IN NS     escalier.maison.net.
\end{minted}

\subsubsection{Résolution inverse d'une zone}

Il ne faut pas oublier de gérer la résolution inverse:
\begin{minted}{bash}
@ IN SOA entree.maison.net. admin.maison.net. (
    2018111601
    604800
    86400
    2419200
    604800
)

@              IN NS     entree.maison.net.
254            IN PTR    entree
\end{minted}
Cela permet de dire que l'IP ``\ldots 254'' appartient au sous-domaine `entree'.

\section{Pare-feux}

\paragraph{Voir la liste des règles}
\begin{minted}{bash}
iptables --list
iptables -t nat --list   # Liste des règles sur la table NAT
\end{minted}

\paragraph{Supprimer les règles}
\begin{minted}{bash}
iptables -F INPUT
iptables -F OUTPUT
iptables -F FORWARD
\end{minted}

Il est conseillé de mettre ces trois lignes en début de script: comme ça les règles précédentes sont supprimées. Cela évite de garder des règles moitié-écrites.

\paragraph{Politique par défaut}
Par défaut, aucun paquet n'est transféré:
\begin{minted}{bash}
iptables -P INPUT DROP
iptables -P OUTPUT DROP
iptables -P FORWARD DROP
\end{minted}

\paragraph{Ajouter une règle}
\begin{minted}{bash}
iptables -A MODE TABLE CONDITIONS -j ACTION
\end{minted}
avec:
\begin{itemize}
\item[\tt MODE] est soit {\tt INPUT}, {\tt OUTPUT} ou {\tt FORWARD}.
\item[\tt TABLE] la table à laquelle on ajoute la règle, {\tt -t nat} ou {\tt -t filter},
\item[\tt ACTION] représente ce qu'il faut faire, {\tt DROP} pour l'abandonner ou {\tt ACCEPT} pour le transmettre (il existe aussi {\tt REJECT} mais est déconseillé),
\item[\tt CONDITIONS] est un ensemble de:
\begin{itemize}
\item[\tt -s] l'IP qui envoie le paquet (sous la forme {\tt 10.1.1.0/24}),
\item[\tt -d] l'IP qui va recevoir le paquet (idem),
\item[\tt -p] le protocole de niveau 4 utilisé: {\tt tcp}, {\tt udp}, {\tt icmp}\ldots
\item[\tt -{}-sport] le port source (on peut utiliser le nom des protocoles connus, {\tt ssh}, {\tt http}\ldots),
\item[\tt -{}-dport] le port destination (idem),
\item[\tt -m conntrack -{}-ctstate ETAT] permet de sélectionner l'état de la connexion ({\tt ETAT} vaut {\tt ESTABLISHED}, {\tt RELATED}, {\tt NEW} ou un mélange (eg. {\tt NEW,ESTABLISHED})),
\end{itemize}
\end{itemize}
\subparagraph{Exemples}
\begin{minted}{bash}
# Accepter les ping (protole ICMP) sur 100.64.0.100
iptables -A INPUT  -p icmp -s 100.64.0.100 -j ACCEPT
iptables -A OUTPUT -p icmp -d 100.64.0.100 -j ACCEPT

# On peut lancer des connexions en HTTP, mais on accepte celles arrivant uniquement
# si ce sont des réponses
iptables -A OUTPUT \
         -p tcp --dport http \
         -j ACCEPT

iptables -A INPUT \
         -p tcp --sport http \
         -m conntrack --ctstate ESTABLISHED \
         -j ACCEPT
\end{minted}

\end{document}
