\documentclass[10pt,a4paper,french]{article}
\usepackage{babel}
\usepackage[utf8]{inputenc}
\usepackage[T1]{fontenc}
\usepackage{../latex/clovisai}

\begin{document}

\title{Probabilités \& Statistiques}
\author{Ivan Canet}
\maketitle

\tableofcontents

\section{Dénombrement}

On distingue deux manières de dénombrer ;

\begin{itemize}
\item Lorsque l'ordre est important, on dit qu'il s'agit d'un \textbf{arrangement},
\item Lorsque l'ordre n'est pas important, on dit qu'il s'agit d'une \textbf{combinaison}.
\end{itemize}

\subsection{Arrangement}

\paragraph{Exemple}
On décide de choisir 5 chevaux parmi 5. Le nombre de possibilités est $20 \times 19 \times 18 \times 17 \times 16$, ce qui est égal à $\frac{20!}{15!}$, dont on peut déduire $\frac{20!}{(20-5)!}$.

\paragraph{Formule \& notation}
On note ``$p$ éléments parmi $n$'' $A_n^p$, et la valeur est $A_n^p = \frac{n!}{(n-p)!}$.

\subsection{Combinaison}

\paragraph{Exemple}
En reprenant l'exemple précédent des 5 chevaux parmi 20; si l'ordre n'est pas important, alors on peut regrouper toutes les possibilités où les mêmes chevaux ont été tirés dans un ordre différent. Il y a 5 chevaux, donc il y a $5!$ cas où les mêmes chevaux ont été tirés. Il faut donc enlever ces cas, ce qui donne un nombre de possibilités de $\frac{A_n^p}{p!}$.

\paragraph{Formule \& notation}
On note ``$p$ éléments parmi $n$'' $n \choose p$, de formule ${n \choose p} = \frac{n!}{p! (n-p)!}$.

\subsection{Triangle de Pascal}

\section{Applications}

\paragraph{Application}
Comme une fonction, mais tous les éléments de l'ensemble de départ doivent absolument avoir une image (on ne peut pas avoir de valeurs interdites).

\paragraph{Application surjective}
Tout élément de l'ensemble d'arrivée a au moins un antécédent.

\paragraph{Loi binomiale}
(cf. exo 10) On reconnaît une loi binomiale de paramètres $n=10$ (nombre de répétions successives et indépendantes) et $P=\frac{1}{12}$ (probabilité d'un succès lors d'une expérience). On l'écrit $B(n,p)$. La variable $X \leadsto B(n,p)$ permet de trouver la probabilité $P(X=k)={n \choose k} p^k (1-p)$.

\paragraph{Exemple loi binomiale}
Lors d'un lancer, j'ai une chance sur trois d'atteindre la cible. Je fais 13 lancers. Quelle est la probabilité que je touche cette cible exactement 7 fois?

On a $n=13$, $p=\frac{1}{3}$ et $k=7$, donc $P(X=7)={13 \choose 7} (\frac{1}{3})^7 (1-\frac{1}{3})^6$.

\paragraph{Probabilité conditionnelle}
$P_B(A)=\frac{P(A \cap B)}{P(B)}$ : ``La probabilité de A quand B est vérifiée''.

\paragraph{Exercice 10}
Désignons par $f_n$ le nombre de manières de jeter une pièce $n$ fois sans que deux piles successifs n'apparaissent.

Soit une partie en $n$ coups au cours de laquelle on n'obtient jamais deux piles consécutives :
\begin{itemize}
\item La partie commence par P, le second lancer doit donc donner F, ensuite toute partie en $n-2$ coups ne donnant pas deux piles consécutives convient; il y en a $f_{n-2}$.
\item La partie commence par F, le second lancer peut être quelconque donc F est suivi d'une quelconque partie en $n-1$ coups ne présentant pas deux piles consécutives. Il y en a $f_{n-1}$.
\end{itemize}

Donc, $f_n = f_{n-1} + f_{n-2}$ avec $f_0 = 1$ et $f_1=2$.

La probabilité d'un événement\footnote{On appelle Cardinal d'un événement (ou plus généralement d'un ensemble) le nombre d'éléments de cet ensemble. Un cardinal peut être fini ou infini.} est $P(A)=\frac{card(A)}{card(\Omega)}$, donc pour l'événement ``Nombre de manières de jeter une pièce $n$ fois sans 2 piles successifs'', l'univers est ``Nombre de manières de jeter une pièce $n$ fois'', ce qui donne $2^n$ possibilités.

On a donc $P(A)=\frac{f_n}{2^n}=\frac{f_{n-1}+f_{n-2}}{2^n}$.

\paragraph{Exercice 1}
Avec remise : $\Omega=\lbrace(R, V), (V, R), (V, B), (B, V), (B, R), (R, B), (R, R), (B, B), (V, V)\rbrace$ donc $card(\Omega)=3^2$ (tirage successif avec remise).

Sans remise : $\Omega=\lbrace(R, V), (V, R), (V, B), (B, V), (B, R), (R, B)\rbrace$ donc $card(\Omega)=3^2-3=3 \times 2$ (tirage successif sans remise).

\paragraph{Exercice 2} 50\% pratique la natation, 55\% pratique le basket, 22.5\% pratique les deux sports.

On en déduit que la probabilité qui pratique la natation et non le basket est $P(N \setminus B) = P(N) - P(N \cup B) = 50 \% - 22.5\% = 27.5 \% $.

De la même manière, la population qui pratique le basket mais pas la natation est $P(B \setminus N)= 55\% - 22.5\% = 32.5\%$.

La population que ne pratique aucun sport est $17.5\%$.

\paragraph{Exercice 3}
5 boules rouges, 6 bleues, 8 vertes ; on tire 3 boules au hasard.

L'univers a une taille de $card(\Omega) = {19 \choose 3}$.

Les événements sont $A_R$ ``toutes les boules sont rouges'', $A_B$ ``toutes les boules sont bleues'' et $A_V$ ``toutes les boules sont vertes''.
Ces trois événements forment une partition de $A$ ``toutes les boules sont de la même couleur'' ; donc 
$P(A) = P(A_R) + P(A_B) + P(A_V)$
avec
$card(A_R)={5 \choose 3}$,
$card(A_B)={8 \choose 3}$ et
$card(A_V)={8 \choose 3}$ ; donc
$P(A)=
\frac
	{{5 \choose 3} + {8 \choose 3} + {8 \choose 3}}
	{{19 \choose 3}}
$.

\paragraph{Exercice 4}
Il y a 3 manières de faire 4 (2+2, 3+1, 1+3) pour un univers d'une taille $6\times6$; donc $P(S=4)=\frac{3}{36}=\frac{1}{12}$.

Pour 10 lancers, il faut que au moins un ai une somme qui vaille 4. On reconnaît une loi binomiale avec $n=10$, $p=\frac{1}{12}$ et $k=$ ``plus de 1'' ; $X$ est le ``nombre de succès''.

$P(X \geq 1) = 1 - P(X=0) = 1 - {0 \choose 10} (\frac{1}{12})^0 (\frac{11}{12})^{10} = 1 - (\frac{11}{12})^{10}$.

\paragraph{Exercice 5}
Il y a $32 \choose 5$ distributions possibles.

\subparagraph{5 cartes de la même couleur}
Le nombre de distributions de 5 cartes de la même couleur est $card(A)={4 \choose 1}\times{8 \choose 5}$. La probabilité de $A$ est donc $P(A)=\frac{4 {8 \choose 5}}{{32 \choose 5}} = \frac{224}{201376}=0.11\%$.

\subparagraph{La dame et le roi de cœur}
La probabilité d'obtenir la dame et le roi de cœur parmi les 5 cartes tirées : $card(A)={1\choose 1}{1\choose 1}{30 \choose 3}$ donc $P(A)=\frac{{30 \choose 3}{1\choose 1}{1\choose 1}}{{32 \choose 5}}$.

\subparagraph{Exactement 3 valets}
La probabilité d'obtenir 3 valets parmi les 5 cartes : $card(A)={4 \choose 3}{28 \choose 2}$. On prend les deux dernières parmi 28 au lieu de parmi 29, parce que la 29\ieme~est le quatrième valet, qu'il ne faut pas prendre.

\subparagraph{Brelan: trois cartes de la même valeur}
La probabilité d'obtenir trois cartes de la même valeur : $card(A)={8 \choose 1} {4 \choose 3} {28 \choose 2}$ parce que l'on choisit la valeur, ensuite il faut en prendre trois de cette valeur (donc parmi 4) puis on prend aléatoirement les 2 dernières parmi ce qu'il reste dans le paquet.

\subparagraph{Carré: quatre cartes de la même valeur}
La probabilité d'obtenir 4 cartes de la même valeur (un carré) : $card(A)={8 \choose 1}{4 \choose 4}{28 \choose 1}$.

\subparagraph{Au moins un brelan}
La probabilité d'obtenir au moins un brelan : $card(A)={8 \choose 1}{4 \choose 3}{29 \choose 2}$.

\subparagraph{Quinte flush}
La probabilité d'obtenir une quinte flush : il y a 4 couleurs, pour chacune on peut faire 4 quintes flush, donc $P(A)=16$.

\subparagraph{Deux paires distinctes}
La probabilité d'obtenir deux paires distinctes : ${8 \choose 1}{4 \choose 2}{7 \choose 1}{4 \choose 2}{24 \choose 1}$ ; on prend une valeur, puis 2 cartes dans cette valeur; on prend une nouvelle valeur (différente donc parmi 7), puis à nouveau 2 cartes dans cette valeur ; on termine avec une carte parmi le reste du paquet.

\paragraph{Exercice 6}
On pose $A$ ``Au moins un 6'' et $B$ ``Les deux sont différents''.

Les cardinaux sont $card(A)=11$ (il n'y a pas d'ordre donc on ne compte pas $(6; 6)$) donc $P(A)=\frac{11}{36}$ et $card(B)=30$ donc $P(B)=\frac{30}{36}$.

On a aussi $card(A \cap B)=10$ donc $P(A \cap B)=\frac{10}{36}$.

Donc $P_B(A) = \frac{P(A \cap B)}{P(B)} = \frac{10 / 36}{30 / 36} = \frac{1}{3}$.

\end{document}