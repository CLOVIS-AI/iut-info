\documentclass[10pt,a4paper]{article}
\usepackage[french]{babel}
\usepackage[utf8]{inputenc}
\usepackage[T1]{fontenc}
\usepackage{../latex/clovisai}

\begin{document}

\title{Probabilités \& Statistiques}
\author{Ivan Canet}
\maketitle

\tableofcontents

\section{Dénombrement}

On distingue deux manières de dénombrer ;

\begin{itemize}
\item Lorsque l'ordre est important, on dit qu'il s'agit d'un \textbf{arrangement},
\item Lorsque l'ordre n'est pas important, on dit qu'il s'agit d'une \textbf{combinaison}.
\end{itemize}

\subsection{Arrangement}

\paragraph{Exemple}
On décide de choisir 5 chevaux parmi 5. Le nombre de possibilités est $20 \times 19 \times 18 \times 17 \times 16$, ce qui est égal à $\frac{20!}{15!}$, dont on peut déduire $\frac{20!}{(20-5)!}$.

\paragraph{Formule \& notation}
On note ``$p$ éléments parmi $n$'' $A_n^p$, et la valeur est $A_n^p = \frac{n!}{(n-p)!}$.

\subsection{Combinaison}

\paragraph{Exemple}
En reprenant l'exemple précédent des 5 chevaux parmi 20; si l'ordre n'est pas important, alors on peut regrouper toutes les possibilités où les mêmes chevaux ont été tirés dans un ordre différent. Il y a 5 chevaux, donc il y a $5!$ cas où les mêmes chevaux ont été tirés. Il faut donc enlever ces cas, ce qui donne un nombre de possibilités de $\frac{A_n^p}{p!}$.

\paragraph{Formule \& notation}
On note ``$p$ éléments parmi $n$'' $n \choose p$, de formule ${n \choose p} = \frac{n!}{p! (n-p)!}$.

\subsection{Triangle de Pascal}

\section{Applications}

\paragraph{Application}
Comme une fonction, mais tous les éléments de l'ensemble de départ doivent absolument avoir une image (on ne peut pas avoir de valeurs interdites).

\paragraph{Application surjective}
Tout élément de l'ensemble d'arrivée a au moins un antécédent.

\end{document}