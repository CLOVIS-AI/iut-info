\documentclass[10pt,a4paper,french]{article}
\usepackage{babel}
\usepackage[utf8]{inputenc}
\usepackage[T1]{fontenc}
\usepackage{../latex/clovisai}

\newacronym{moa}{MOA}{Maîtrise d'Ouvrage}
\newacronym{moe}{MOE}{Maîtrise d'Œuvre}

\makeindex
\makeglossaries

\begin{document}

\title{Gestion de projet}
\author{Ivan Canet}
\maketitle

\begin{abstract} % ARTICLE ONLY
Ceci est l'introduction du document.
\end{abstract}

\tableofcontents

\part{Charte de Projet}

\section{Qu'est-ce qu'un projet ?}

\begin{cquote}{Norme ISO 10006: 2003}
Un projet est un processus unique, qui consiste en un ensemble d'activités coordonnées et maîtrisées, comportant des dates de début et de fin, entreprises dans le but d'attendre un object précis, sous des contraintes de délais, de coûts et de ressources.
\end{cquote}

\begin{cquote}{Lexique de gestion Dalloz}
En sciences de gestion, le projet désigne une forme d'organisation de la production mise en œuvre pour la conception et la réalisation ... de produits innovants. Un groupe projet est constitué ... d'une équipe pluridisciplinaire mise sous la responsabilité d'un chef de projet. Il permet de travailler à partir des compétences acquises dans les différents secteurs de l'entreprise.
\end{cquote}

Les critères justifiant un processus de gestion sont:
\begin{itemize}
\item La nature du projet à réaliser
\item La complexité technique du produit attendu
\item Le périmètre à couvrir: les fonctions couvertes dans l'entreprise
\item La caractère plus ou moins innovant du produit à réaliser
\end{itemize}

Il faut bien faire la différence entre la gestion du projet et la gestion du produit.

Les étapes de création du produit sont:
\begin{enumerate}
\item Analyse: organiser l'analyse, rédaction du cahier des charges
\item Conception: rédaction du cahier de spécifications, planification du développement
\item Développement
\item Tests: permet de valider la conception
\item Recette: permet de valider la conformité aux demandes du client (l'analyse)
\end{enumerate}

Les attentes des parties prenantes diffèrent selon que :
\begin{itemize}
\item La \gls{moa} et la \gls{moe} appartiennent à la même organisation ?
\item Le client a-t-il des exigences/habitudes ?
\item Le client est-il familier avec le projet ?
\item Le secteur d'activité est-il soumis à des normes ?
\end{itemize}

On fait la différence entre les normes, les standards et les bonnes pratiques :
\begin{description}
\item[Norme\mainentry{Norme}] Les normes sont des règles décidées par des organismes connus (eg. l'AFNOR en France, ou l'ISO). Elles sont facultatives, mais vivement conseillées.
\item[Standard\mainentry{Standard}] Une habitude que la majorité utilise (eg. les prises USB)
\item[Bonnes pratiques\mainentry{Bonnes pratiques}] Recommandé par de grandes entreprises
\item[Certification\mainentry{Certification}] Preuve que l'on maîtrise une bonne pratique, délivrée par des organisation (eg. la certification PMI pour les employés)
\end{description}

\section{Charte de projet}

Le but de la charte de projet est:
\begin{itemize}
\item Autoriser formellement un projet ou une nouvelle phase de projet
\item Identifier le responsable du projet ainsi que les responsabilités et autorités appropriées de celui-ci
\item Documenter les besoins commerciaux, les objectifs, les résultats attendus et les aspects économiques du projet.
\end{itemize}

\begin{cquote}{NF ISO 21500}
La charte du projet établit un lien entre le projet et les objectifs stratégiques de l'organisation, et il convient d'identifier les éventuels mandats, obligations, hypothèses et contraintes appropriés.
\end{cquote}

Il faut en premier se demander qui sont les parties prenantes, on peut le savoir en organisant des groupes de discussions qualitatifs:
\begin{itemize}
\item Qui est impacté par le projet?
\item Qui est impacté par les résultats du projet?
\item Leur rôle?
\item Leurs attentes et intérêts?
\item Leur pouvoir d'influence?
\item Leurs degré de connaissance du projet?
\item Leurs rapports entre elles?
\end{itemize}

Pour chaque tâche, on peut alors choisir quelles parties prenantes sont ``responsable'' de tâches, ``réalisatrices'', ``consultées'' ou ``informées''.

Il ne faut pas oublier l'utilisateur final, qui est toujours une partie prenante!

\begin{cquote}{PMBOK}
Un travail d'analyse et de partage d'idées est requis pour construire un registre complet et de définir la position de chaque partie prenante vis-à-vis du projet: partisan, opposant, neutre, indifférent, fortement impliqué, ou fortement opposé.

Le registre des parties prenantes est un tableau vivant qui est modifiable tout au long du projet. Il va nous servir à gérer les communications, l'implication des parties prenantes et la conduite du changement.
\end{cquote}

Ce qu'on retrouve dans une charte de projet (elle doit toujours faire moins de 10 pages):
\begin{itemize}
\item Introduction
\item Documents de référence
\item Présentation du projet
\item Équipe de projet et les parties prenantes
\item Attentes auxquelles doit répondre le projet
\item Objectifs généraux du projet et des bénéfices attendus
\item Critères de succès du projet
\item Risques
\item Hypothèses (eg. un exemple d'interface)
\item Contraintes
\item Exclusions
\item Facteurs clé de succès
\item Organisation du projet
\item Registre des parties prenantes (en annexe)
\end{itemize}

\subsection{Objectifs}

\begin{cquote}{}
Aux origines du projet, il y a un besoin initial avéré qui permet d'assigner au projet des objectifs à atteindre et des indicateurs de réussite.
\end{cquote}

Il faut reformuler le besoin, identifier les objectifs (stratégique, performance, économique, délai) et préciser les indicateurs qui permettront de s'assurer de la réussite du projet.

On essaie de regrouper les objectifs en catégories (Stratégie, Économie, Performances, Délais).

On donne un indicateur à chaque objectif (= une manière de le quantifier). Par exemple, le nombre de personnes qui ont rejoint le site, \ldots

\part{Plan de projet}

\begin{cquote}{NF ISO 21500}
L'objectif du plan de projet est de renseigner la raison pour laquelle le projet est entrepris, ce qui va être fourni, ce que cela coûtera, et la manière dont il sera mis en œuvre, maîtrisé et clos.
\end{cquote}

Pour cela, on va le faire en 5 parties:
\begin{enumerate}
\item Les objectifs et le calendrier du projet,
\item Les risques,
\item Le plan de communication et la gestion du management
\item Le budget,
\item Le pilotage.
\end{enumerate}

\section{Les objectifs et le calendrier du projet}

Le point de départ est le besoin du client. On commence par réaliser les spécifications fonctionnelles pour structurer l'œuvre.

\subsection{Compréhension et analyse du besoin}

On peut utiliser les techniques suivantes:
\begin{itemize}
\item Quoi, où, quand, comment, pourquoi, combien,
\item Méthodes des 9 cases,
\item Diagrammes Use-Case en UML.
\end{itemize}

\subsubsection{Techniques des 9 cases}

On commence par identifier les besoins en largeur (les thèmes qui seront utilisés) puis en profondeur (les détailler).

\begin{tabular}{c|ccc}
& Quel est le problème? & Qui est impacté? & Visualiser la solution \\
\hline
Question ouverte & 1 & 4 & 7 \\
Combien, quand, où & 2 & 5 & 8 \\
Si je comprends bien... & 3 & 6 & 9 \\
\end{tabular}

\subsubsection{Analyse fonctionnelle}

On peut aussi utiliser l'analyse fonctionnelle (voir la norme NF X50-151).

Il s'agit de répondre aux questions suivantes:
\begin{itemize}
\item À qui le produit rend-il service?
\item Sur qui ou quoi le produit agit-il?
\item Dans quel but?
\end{itemize}

On recherche ensuite les fonctions de service (``Quelles actions le produit doit-il réaliser pour répondre à chaque élément du besoin d'un utilisateur'') et les fonctions d'adaptations (``À quelles contraintes sont soumises les fonctions de services pour pouvoir interagir avec l'environnement du produit (les interacteurs)?'').

Il ne faut pas oublier la maintenance.

On fait ensuite la liste des résultats à obtenir.

\subsubsection{Hiérarchisation des besoins}

On examine le bénéfice attendu, le coût de réalisation et les risques.

\paragraph{Loi de Pareto}
Dans un cas où ``20\% des produits représentent 80\% du chiffre d'affaire'', on se concentre sur ces 20\%. Cela permet de réduire les coups.

\paragraph{Matrice de hiérarchisation en fonction des risques et valeurs}
On classe les objectifs selon leur ``risque'' et selon leur ``valeur'' ; on a donc une matrice de 4 zones: ``prioritaire'', ``secondaire'', ``à éviter'', ``reporté''.

\paragraph{MoSCoW}
\begin{itemize}
\item `Must have' : fonctions indispensables
\item `Should have' : fonctions souhaitables mais non-prioritaires
\item `Could have' : fonctions possibles
\item `Should have but won't have' : fonctions éliminées, reportées 
\end{itemize}

\paragraph{Matrice de Kano}
On sépare les fonctions `obligatoires' (le client n'est pas content si elles ne sont pas là), `attractives' (le client ne se rend pas compte qu'elles ne sont pas là, mais elles ont un gros bonus sur l'impression du client), `proportionnelles' (standard).

\subsection{Structuration du projet}

On sépare le produit en des sous-tâches: par exemple, l'application, la documentation, le déploiement. On peut ensuite séparer l'application entre le client et le serveur, puis entre les différents écrans.

On peut ensuite créer les tâches correspondant à chacune des feuilles.

On décide alors qui est responsable de quelle tâche pour la réalisation, le contrôle, etc.

On peut alors séparer les tâches en sous-tâches plus petites, puis on calcule les charges, on continue avec le PERT (pour repérer le chemin critique et les tâches parallélisables) puis le Gantt.

\section{Analyse des risques}

\begin{cquote}{AFITEP, 2000}
Un risque est la possibilité qu'un projet ne s'exécute pas conformément aux prévisions de dates d'achèvement, de coûts de spécifications, ces écarts par rapport aux prévisions étant considérés comme difficilement acceptables, voire inacceptables.
\end{cquote}

\begin{cquote}{ISO 21500}
Les risques ayant un impact négatif potentiel sur le projet sont appelés ``menaces'' et les risques ayant un impact positif potentiel sont appelés ``opportunités''.
\end{cquote}

Les risques peuvent venir de 3 sources:
\begin{description}
\item[Incertitude] Insuffisance d'information qui empêche de prendre des décisions sûres
\item[Imprévu] Événements imprévisibles
\end{description}

Il faut ensuite pouvoir traiter les risques (transférer une tâche à quelqu'un d'autre, \dots).

Durant tout le projet, il faut suivre les risques et agir pour les limiter.

\subsection{Identification}

\paragraph{Matrice SWOT}
On fait un tableau:
\begin{tabular}{c|cc}
& + & - \\
\hline
Projet & \textit{Forces} & \textit{Faiblesses} \\
Environnement & \textit{Opportunités} & \textit{Menaces}
\end{tabular}

\paragraph{Diagramme d'Ishikawa}

\paragraph{Typologie de l'AFITEP}
\begin{tabular}{c|ccc}
Objectif & Définition & Exemple & Risques \\
\hline
Stratégie & & & \\
Efficience & & &
\end{tabular}

\paragraph{Profil de risque}
\begin{tabular}{c|c}
Nature du risque & Degré de risque (1\ldots 5) \\
\hline
Taille du projet & \\
Difficulté technique & \\
Degré d'intégration & \\
Configuration organisationnelle & \\
Changement pour les utilisateurs\footnote{Les changements représentent ce qui est modifié pour les utilisateurs au cours du projet, après sa réalisation (modifications de maintenance, \ldots)} & \\
Instabilité de l'équipe & \\
\end{tabular}

\subsection{Analyse}

Les critères sont:
\begin{description}
\item[Gravité] L'impact sur les délais et les coûts
\item[Probabilité d'occurrence]
\end{description}

\begin{tabular}{c|ccc}
& Grave & Moyen & Pas grave \\
\hline
Probabilité forte & Risque inacceptable & Élevé & Modéré \\
Probabilité moyenne & Élevé & Modéré & Faible \\
Probabilité faible & Modéré & Faible & Faible
\end{tabular}

\paragraph{Actions pour réduire les risques}
\begin{itemize}
\item Limiter l'incertitude par plus de spécifications, prototypage, maquettage, consultation d'experts
\item Renforcer les ressources humaines (Quantité, qualité)
\item Mettre en place des actions pour diminuer le risque
\item Contourner le risque (acheter un logiciel, \ldots)
\end{itemize}

\section{Communication \& management}

Le plan de communication permet de se coordonner, de se convaincre et de réduire les risques. Toutes les parties prenantes sont concernées, dont les membres de l'équipe elle-même.

Pour chaque type de communication, on va établir une cible (parties prenantes), des objectifs, la description du message, un média (email, Slack, réunion\ldots) et la périodicité (tous les jours\ldots).

\section{Budget}

Le but est d'estimer le budget et les coûts du projet.

Dans un projet de développement, les plus gros coûts sont les coûts RH, mais il ne faut pas oublier les coûts achat (matériel, logiciel, sous-traitance\ldots), les frais divers (communication, risques, déplacements, formations\ldots). On ajoute souvent une réserve pour aléas de 5--10\% en cas d'imprévus ou de dépassement.

\subsection{Coûts Ressources Humaines}
\[ \text{Coût total prévisionnel} = \text{Quantité de travail planifié (j/h)} \times \text{Coût standard} \]

Le coût standard est formé par le coût horaire de l'employé (différent selon les qualifications), mais il ne faut pas oublier les charges sociales, mais aussi l'équipement, les congés\ldots

\paragraph{Travail en régie}
permet à une entreprise d'avoir des employés d'une ESN sur place, ce qui est beaucoup plus flexible.

On représente le budget sous forme de tableau au court du temps.

\section{Pilotage}
Il faut surveiller les écarts par rapport aux différents diagrammes préparés aux parties précédentes.

\appendix % Annexes, ARTICLE & BOOK
\part{Annexes}
%\bibliography{•} % THE .BIB FILE HERE, WITHOUT THE EXTENSION
\printglossaries
\printindex

\end{document}
