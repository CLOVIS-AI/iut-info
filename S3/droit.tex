\documentclass[10pt,a4paper,french]{article}
\usepackage{babel}
\usepackage[utf8]{inputenc}
\usepackage[T1]{fontenc}
\usepackage{../latex/clovisai}

\makeindex
\makeglossaries

\begin{document}

\title{Droit S3}
\author{Ivan Canet}
\maketitle

%\begin{abstract} % ARTICLE ONLY
%Ceci est l'introduction du document.
%\end{abstract}

\tableofcontents

\part{Droit Contractuel et Informatique}

On fait la différence entre les rapports contractuels\mainentry{Rapports!contractuels} et extra-contractuels\mainentry{Rapports!extra-contractuels} (aussi appelés\mainentry{Rapports!délictuels} délictuels).

\section{Droit des contrats}

\subsection{Principes généraux}

Les lois sur les contrats sont dans le code\mainentry{Code civil} civil, et peuvent être complétées dans d'autres codes (du travail, du commerce\ldots). Les différentes lois sont disponibles sur \hrefu{legifrance.gouv.fr}{legifrance}.

\paragraph{Bonne-foi\mainentry{Bonne-foi}}

\begin{cquote}{Article 1103 du Code Civil\entry{Code civil}}
Les contrats doivent être formés et exécutés de bonne-foi.
\end{cquote}

Un contrat de mauvaise foi (eg. promettre quelque chose d'impossible) est annulable légalement.

\paragraph{Liberté contractuelle\mainentry{Liberté contractuelle}}
Tout le monde choisit son co-contractant : personne ne peut être forcé à former un contrat. Attention, cela reste cadré, par exemple un vendeur ne peut pas refuser de vendre pour raisons personnelles (lois contre la discrimination).

Un individu a aussi le droit d'écrire le contrat comme il le souhaite à condition de respecter toutes les lois (droit de la personnalité, pas d'atteinte à l'ordre public).

\subsection{Conditions de formation de contrat}

\subsubsection{Préalable : les pourparlers\mainentry{Pourparlers}}
Le Code Civil donne une liberté totale sur les pourparlers (négociations) et leur déroulement.

Un contracteur peut demander des dommages et intérêts lorsque les négociations sont rompues de mauvaise foi (responsabilité délictuelle).

\subsubsection{Les conditions}

Quand un contrat est scellé, chaque partie devient créancier\entry{Créancier} et\entry{Débiteur}débiteur.

Une offre de contrat doit comporter les éléments essentiels. À ce moment-là, une autre partie peut accepter le contrat : il entre alors immédiatement en action, de manière non-négociable.

Les conditions doivent donc être ajoutées au contrat avant l'échange de\entry{Consentement}consentement. Pour l'achat à distance, un droit de rétractation légal existe.

\paragraph{Consentement\mainentry{Consentement}}
Accord de deux ou plusieurs volontés en vue de créer les effets de droit. \\
Le silence ne vaut pas acceptation (sauf dans certains cas de professionnel à professionnel). \\
Trois vices\mainentry{Consentement!Vices du} du consentement peuvent annuler le contrat : l'erreur\mainentry{Erreur} (le contrat a été mal compris), le dol\mainentry{Dol} ou la tromperie\mainentry{Tromperie} (manœuvres, mensonges ou dissimulation intentionnelle) et la violence\mainentry{Violence} (physique, morale, économique).

\paragraph{Information\mainentry{Obligations!d'information} et conseil}
Un professionnel doit donner des informations et conseils en fonction de l'état de l'art : par exemple, ne pas promettre quelque chose d'impossible. Le prestataire doit être capable de prouver qu'il a donné les bons conseils.

\subsection{Contenu du contrat}

\subsubsection{Engagements\mainentry{Engagements}}
On peut écrire tout ce que la loi n'interdit pas. On retrouve deux types d'engagements mutuels:
\begin{itemize}
\item Engagements essentiels\mainentry{Engagements!essentiels}: Livrer le logiciel, fournir la documentation
\item Engagements accessoires\mainentry{Engagements!accessoires}: 
\end{itemize}

On ne peut pas contrevenir aux obligations essentielles:

\begin{cquote}{•}
Toute clause ou obligation qui prive de substance l'obligation essentielle du contrat est réputée non-écrite.
\end{cquote}
Par exemple, les contrats de livraison Chronopost ajoutaient une clause ``nous ne sommes pas responsables des retards''.

\subsubsection{Obligations\mainentry{Obligations}}
\paragraph{Obligation de moyens\mainentry{Obligations!de moyen}}
On s'engage à obtenir un résultat, sans le garantir (par exemple on peut avoir un retard).

\paragraph{Obligation de résultats\mainentry{Obligations!de résultats}}
On s'engage à obtenir un résultat, sans aucune différence.

\begin{itemize}
\item En\entry{B2C}B2C, il s'agit d'obligation de résultats;
\item En\entry{B2B}B2B, on peut choisir dans le contrat. Si ce n'est pas précisé, on dit qu'une prestation soumise à des aléas ou une haute technicité est soumise à une obligation de moyens, et une prestation peu technique ou peu aléatoire est soumise à une obligation de résultats.
\end{itemize}

\subsubsection{Clauses abusives}
Une clause abusive est une clause qui crée un déséquilibre significatif entre les droits et les parties (eg. date de livraison incertaine mais date de paiement définie).

Une clause abusive est définie comme non-écrite.

\subsection{Forme du contrat\mainentry{Contrat!Formes du}}
En droit français, le contrat est dit consensuel (il relève du consentement) : à chaque fois que l'on échange avec quelqu'un, on forme un contrat. La forme principale est donc la forme orale\mainentry{Contrat!oral}.

Un contrat écrit\mainentry{Contrat!écrit} est considéré comme une preuve certaine\entry{Preuve!certaine} s'il est daté, signé, nommé précisément, et en le même nombre d'exemplaires que de parties (3 parties $\to$ 3 exemplaires).

Un enregistrement compte comme un ``commencement de preuve''\entry{Preuve!Commencement de}: le juge a un pouvoir d'interprétation et peut annuler le contrat.

Pour les contrats\mainentry{Contrat!électronique}électroniques, la loi impose un processus en trois étapes : le fournisseur donne au consommateur la possibilité de sélectionner l'offre (le panier), puis confirmer l'acceptation de l'offre (paiement) ; puis le fournisseur doit reconfirmer que le contrat a été passé (mail de confirmation).

\subsection{Sanctions des conditions de formation}
Dans un cas d'un objet de contrat illicite, le contrat est annulé de manière rétroactive.

Le client est alors remboursé de tous les frais, mais il est impossible d'annuler le travail éventuel de prestataire. Le prestataire est donc toujours perdant.

\subsection{Exécution des contrats}

\paragraph{Force obligatoire\mainentry{Contrat!Force obligatoire du}}
Pour les parties, le contrat a valeur de loi. Pour annuler un contrat (où aucune réserve n'a été précisée), il faut l'accord des autres parties.

\paragraph{Effet relatif\mainentry{Contrat!Effet relatif du}}
Le contrat n'a d'effet qu'entre les parties, il ne peut préjudicier un tiers (mais parfois, un tiers peut en bénéficier). \\
Par exemple, on ne peut pas passer un contrat pour récupérer la propriété de quelqu'un, mais on peut passer un contrat pour faire un don à quelqu'un.

\paragraph{Caducité}
Lorsque deux contrats sont liés, la disparition de l'un entraîne la disparition de l'autre.

Souvent, il faut donc prévoir une clause dans le contrat à propos de ce qu'il arrive dans un cas de caducité (payer une partie du prix\ldots).

\paragraph{Imprévision}
En cas de prestation non conforme à ce qui a été convenu, le client peut demander une réduction du prix.

On peut convenir dans le contrat que l'imprévision n'a pas de conséquences : le contrat n'est pas modifié.

\paragraph{Interprétation}
Un contrat s'interprète toujours en faveur du client.

\subsection{Inexécution des contrats}

\paragraph{Exécution forcée\mainentry{Exécution!forcée}}
On peut forcer le prestataire à exécuter le contrat. \\
On peut aussi demander à un autre prestataire de réaliser la prestation, mais de faire payer la facture au premier.

\paragraph{Cas de force majeure\mainentry{Force majeure}}
Quand le prestataire est face à un cas de force majeure, le contrat est annulé et le prestataire n'est obligé à rien. \\
Un cas de force majeure correspond à un événement irrésistible, imprévisible et extérieur. \\
Au bout d'un mois, il est possible d'annuler le contrat.

\paragraph{Exception d'inexécution}
Il s'agit de conditionner son exécution à l'exécution de l'autre : ``œil pour œil, dent pour dent''. \\
Par exemple: si on n'est pas livré, on ne paie pas. Si on sait que l'autre ne pourra pas payer, on ne livre pas.

\paragraph{Réduction du prix}
Si le contrat n'est pas totalement réalisé, le client peut demander une réduction du prix.

\paragraph{Résolution\mainentry{Contrat!Résolution du}ou résiliation\mainentry{Contrat!Résiliation du}du contrat}
Destruction du contrat au tort de l'autre partie (demandée à un juge).

\paragraph{Responsabilité contractuelle\mainentry{Responsabilité contractuelle}}
Demander au co-contractant des dommages et intérêts, compte tenu du préjudice né de l'inexécution contractuelle.

\section{Contrats Informatique}

\subsection{Phase pré-contractuelle}
Le client envoie une étude de besoin\entry{Besoin!Étude de} ou une expression du\entry{Besoin!Expression de} besoin, auquel on répond par les\entry{Spécificités} spécificités. Ensemble, ces documents représentent le cahier des\mainentry{Cahier des charges} charges.

On établie un contrat, qui contient le cahier des charges comme annexe. Il faut donc écrire un cahier des charges très détaillé.

\subsection{Types de contrats}

\subsubsection{Contrat de sous-traitance}
Si un client souhaite quelque chose que vous ne réalisez pas (par exemple de l'hébergement), vous pouvez proposer au client de prendre le contrat d'hébergement vous-même et de lui facturer.

\subsubsection{Contrat de co-traitance}
Cela vous permet d'avoir un contrat avec plusieurs prestataires (par exemple, les différents départements d'une grande entreprise).

\subsection{Exécution du contrat}
Les contrats informatique ont une exécution segmentée: on commence par l'analyse fonctionnelle, l'analyse organique, la programmation, puis l'intégration et la livraison.

Dans le cahier des charges, on inclue le ``calendrier des livrables'', qui est donc inclus dans le contrat. On peut associer chaque étape à un acompte.

Pendant la période de recette, le client teste le logiciel.

\part{Bases de données}

\part{Réseaux et Internet}

\appendix % Annexes, ARTICLE & BOOK

%\bibliography{•} % THE .BIB FILE HERE, WITHOUT THE EXTENSION
\printindex
%\cprintglossaries

\end{document}
