\documentclass[10pt,a4paper,french]{article}
\usepackage{babel}
\usepackage[utf8]{inputenc}
\usepackage[T1]{fontenc}
\usepackage{../latex/clovisai}

\makeindex
\makeglossaries

\begin{document}

\title{Droit S3}
\author{Ivan Canet}
\maketitle

%\begin{abstract} % ARTICLE ONLY
%Ceci est l'introduction du document.
%\end{abstract}

\tableofcontents

\part{Droit Contractuel et Informatique}

On fait la différence entre les rapports contractuels et extra-contractuels (aussi appelés délictuels).

\section{Droit des contrats}

\subsection{Principes généraux}

Les lois sur les contrats sont dans le code civil, et peuvent être complétées dans d'autres codes (du travail, du commerce\ldots).

\paragraph{Bonne-foi}

\begin{cquote}{Article 1103 du Code Civil}
Les contrats doivent être formés et exécutés de bonne-foi.
\end{cquote}

Un contrat de mauvaise foi (eg. promettre quelque chose d'impossible) est annulable légalement.

\paragraph{Liberté contractuelle}
Tout le monde choisit son co-contractant : personne ne peut être forcé à former un contrat. Attention, cela reste cadré, par exemple un vendeur ne peut pas refuser de vendre pour raisons personnelles (lois contre la discrimination).

Un individu a aussi le droit d'écrire le contrat comme il le souhaite à condition de respecter toutes les lois (droit de la personnalité, pas d'atteinte à l'ordre public).

\subsection{Conditions de formation de contrat}

\subsubsection{Préalable : les pourparlers}
Le Code Civil donne une liberté totale sur les pourparlers (négociations) et leur déroulement.

Un contracteur peut demander des dommages et intérêts lorsque les négociations sont rompues de mauvaise foi (responsabilité délictuelle).

\subsubsection{Les conditions}

Quand un contrat est scellé, chaque partie devient créancier et débiteur.

Une offre de contrat doit comporter les éléments essentiels. À ce moment-là, une autre partie peut accepter le contrat : il entre alors immédiatement en action, de manière non-négociable.

Les conditions doivent donc être ajoutées au contrat avant l'échange de consentement. Pour l'achat à distance, un droit de rétractation légal existe.

\paragraph{Consentement}
Accord de deux ou plusieurs volontés en vue de créer les effets de droit. \\
Le silence ne vaut pas acceptation (sauf dans certains cas de professionnel à professionnel). \\
3 vices du consentement peuvent annuler le contrat : l'erreur (le contrat a été mal compris), le dol ou la tromperie (manœuvres, mensonges ou dissimulation intentionnelle) et la violence (physique, morale, économique).

\paragraph{Information et conseil}
Un professionnel doit donner des informations et conseils en fonction de l'état de l'art : par exemple, ne pas promettre quelque chose d'impossible. Le prestataire doit être capable de prouver qu'il a donné les bons conseils.

\section{Contrats Informatique}

\part{Bases de données}

\part{Réseaux et Internet}

\appendix % Annexes, ARTICLE & BOOK

%\bibliography{•} % THE .BIB FILE HERE, WITHOUT THE EXTENSION
%\cprintindex
%\cprintglossaries

\end{document}
