\documentclass[10pt,a4paper,french]{article}
\usepackage{babel}
\usepackage[utf8]{inputenc}
\usepackage[T1]{fontenc}
\usepackage{../latex/clovisai}

\makeindex
\makeglossaries

\begin{document}

\title{Droit S3}
\author{Ivan Canet}
\maketitle

%\begin{abstract} % ARTICLE ONLY
%Ceci est l'introduction du document.
%\end{abstract}

\tableofcontents

\part{Droit Contractuel et Informatique}

On fait la différence entre les rapports contractuels et extra-contractuels (aussi appelés délictuels).

\section{Droit des contrats}

\subsection{Principes généraux}

Les lois sur les contrats sont dans le code civil, et peuvent être complétées dans d'autres codes (du travail, du commerce\ldots). Les différentes lois sont disponibles sur \hrefu{legifrance.gouv.fr}{legifrance}.

\paragraph{Bonne-foi}

\begin{cquote}{Article 1103 du Code Civil}
Les contrats doivent être formés et exécutés de bonne-foi.
\end{cquote}

Un contrat de mauvaise foi (eg. promettre quelque chose d'impossible) est annulable légalement.

\paragraph{Liberté contractuelle}
Tout le monde choisit son co-contractant : personne ne peut être forcé à former un contrat. Attention, cela reste cadré, par exemple un vendeur ne peut pas refuser de vendre pour raisons personnelles (lois contre la discrimination).

Un individu a aussi le droit d'écrire le contrat comme il le souhaite à condition de respecter toutes les lois (droit de la personnalité, pas d'atteinte à l'ordre public).

\subsection{Conditions de formation de contrat}

\subsubsection{Préalable : les pourparlers}
Le Code Civil donne une liberté totale sur les pourparlers (négociations) et leur déroulement.

Un contracteur peut demander des dommages et intérêts lorsque les négociations sont rompues de mauvaise foi (responsabilité délictuelle).

\subsubsection{Les conditions}

Quand un contrat est scellé, chaque partie devient créancier et débiteur.

Une offre de contrat doit comporter les éléments essentiels. À ce moment-là, une autre partie peut accepter le contrat : il entre alors immédiatement en action, de manière non-négociable.

Les conditions doivent donc être ajoutées au contrat avant l'échange de consentement. Pour l'achat à distance, un droit de rétractation légal existe.

\paragraph{Consentement}
Accord de deux ou plusieurs volontés en vue de créer les effets de droit. \\
Le silence ne vaut pas acceptation (sauf dans certains cas de professionnel à professionnel). \\
Trois vices du consentement peuvent annuler le contrat : l'erreur (le contrat a été mal compris), le dol ou la tromperie (manœuvres, mensonges ou dissimulation intentionnelle) et la violence (physique, morale, économique).

\paragraph{Information et conseil}
Un professionnel doit donner des informations et conseils en fonction de l'état de l'art : par exemple, ne pas promettre quelque chose d'impossible. Le prestataire doit être capable de prouver qu'il a donné les bons conseils.

\subsection{Contenu du contrat}

\subsubsection{Engagements}
On peut écrire tout ce que la loi n'interdit pas. On retrouve deux types d'engagements mutuels:
\begin{itemize}
\item Engagements essentiels: Livrer le logiciel, fournir la documentation
\item Engagements accessoires: 
\end{itemize}

On ne peut pas contrevenir aux obligations essentielles:

\begin{cquote}{•}
Toute clause ou obligation qui prive de substance l'obligation essentielle du contrat est réputée non-écrite.
\end{cquote}
Par exemple, les contrats de livraison Chronopost ajoutaient une clause ``nous ne sommes pas responsables des retards''.

\subsubsection{Obligations}
\paragraph{Obligation de moyens}
On s'engage à obtenir un résultat, sans le garantir (par exemple on peut avoir un retard).

\paragraph{Obligation de résultats}
On s'engage à obtenir un résultat, sans aucune différence.

\begin{itemize}
\item En B2C, il s'agit d'obligation de résultats;
\item En B2B, on peut choisir dans le contrat. Si ce n'est pas précisé, on dit qu'une prestation soumise à des aléas ou une haute technicité est soumise à une obligation de moyens, et une prestation peu technique ou peu aléatoire est soumise à une obligation de résultats.
\end{itemize}

\subsubsection{Clauses abusives}
Une clause abusive est une clause qui crée un déséquilibre significatif entre les droits et les parties (eg. date de livraison incertaine mais date de paiement définie).

Une clause abusive est définie comme non-écrite.

\subsection{Forme du contrat}
En droit français, le contrat est dit consensuel (il relève du consentement) : à chaque fois que l'on échange avec quelqu'un, on forme un contrat. La forme principale est donc la forme orale.

Un contrat écrit est considéré comme une preuve certaine s'il est daté, signé, nommé précisément, et en le même nombre d'exemplaires que de parties (3 parties $\to$ 3 exemplaires).

Un enregistrement compte comme un ``commencement de preuve'': le juge a un pouvoir d'interprétation et peut annuler le contrat.

\section{Contrats Informatique}

\part{Bases de données}

\part{Réseaux et Internet}

\appendix % Annexes, ARTICLE & BOOK

%\bibliography{•} % THE .BIB FILE HERE, WITHOUT THE EXTENSION
%\cprintindex
%\cprintglossaries

\end{document}
