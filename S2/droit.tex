\documentclass[10pt,a4paper]{article}
\usepackage[french]{babel}
\usepackage[utf8]{inputenc}
\usepackage[T1]{fontenc}
\usepackage{../latex/clovisai}

\begin{document}

\title{Cours de droit, S2}
\author{Ivan Canet}
\maketitle
\tableofcontents

\part{Introduction au droit}

\section{La règle de droit}

\subsection{La finalité de la règle de droit}

La règle de droit, aussi appelée loi, a comme but d'organiser la vie en société et les relations entre les membres qui la composent. Les buts principaux sont d'assurer la sécurité des personnes (le code pénal), la protection des salariés (le code du travail) et la sécurité des biens (pénalisation du vol), mais aussi l'organisation économique (droit des marchés, droit du commerce), le gouvernement, l'organisation sociale.

\subsection{Caractère général}

La règle de droit est une règle de conduite, qui s'applique à tout le territoire pour tout ce qui s'y produit. Elle est impersonnelle (elle n'admet pas de cas particuliers).

La règle de droit est un commandement qui doit être respecté. Il en existe deux sortes; les règles impératives (qui ordonnent ou interdisent une conduite), et les règles supplétives.

\subsection{Droit objectif et droit subjectif}

Le droit objectif est un ensemble de règles de droit régissant la vie en société, les rapports entre particuliers, le fonctionnement de l'État, son rapport avec les citoyens. Le droit objectif pose parfois des interdictions.

Le droit subjectif est une prérogative possédée à l'égard de biens ou de personnes. Il est reconnu par le droit objectif qui tachera d'en déterminer la nature et l'étendue.

\section{Les branches du droit}

Les branches du droit sont un classement des règles de droit en fonction de l'objet ou du domaine.

\subsection{Droit privé et droit public}

Le droit public définit les rapports entre l'État et les particuliers. Il est séparé entre le droit constitutionnel et le droit administratif.

Le droit privé définit les rapports entre particuliers, il est séparé entre le droit civil (rapports entre personnes privés en l'absence de règles spécifiques) et le droit commercial (le statut des commerçants et des sociétés).

On retrouve aussi le droit mixte, c'est-à-dire le droit pénal, qui a pour rôle de proposer la peine.

\subsection{Droit national et international}

Le droit national réglemente les rapports sociaux à l'intérieur d'un État et le droit international réglemente les relations entre plusieurs États, gérés par des traités (accords conclus entre différents États).

Le droit international est divisé en trois branches:
\begin{itemize}
\item Le droit communautaire (eg. l'UE),
\item Le droit international public (les rapports entre les États et les compétences des organisations internationales),
\item Le droit international privé (rapports entre ressortissants qui relèvent d'États différents).
\end{itemize}

\section{Les sources du droit}

Dans le droit ancien, les coutumes servaient de règles, avec les ordonnances du Roi et le droit canonique (l'Église).

\subsection{La constitution}

La constitution peut être définie comme un ensemble de règles fondant l'autorité étatique, organisant ses institutions, lui donnant ses pouvoirs et imposant des limitations. Elle vient garantir les libertés des citoyens. Notre constitution a été écrite le 4 octobre 1958 et a été modifiée plusieurs fois (par exemple lors du passage du septennat au quinquennat).

\subsection{Les autres sources du droit}

\subsubsection{La loi}

Les lois interviennent dans tous les domaines du pouvoir législatif, énumérées par l'article 34 de la constitution. Une loi est votée par le parlement (l'Assemblée Nationale et le Sénat). Le pouvoir exécutif peut avoir un pouvoir normatif: règlements, décrets, arrêtés. C'est la constitution qui définit les domaines de compétences.

\subsubsection{La jurisprudence}

C'est l'ensemble des décisions rendues par des juridictions (les tribunaux), soit dans une matière, soit dans une branche du droit. Elle désigne une solution donnée par les juridictions à un problème donné, dont on pourra s'inspirer pour les problèmes futurs similaires.

\subsubsection{Hiérarchie des normes}

Pour empêcher une contradiction entre différentes sources de loi, le 30 octobre 1998, il a été déclaré que la constitution est supérieure aux traités -- mais l'Union Européenne considère que le droit communautaire est supérieur.

\part{Personnes et patrimoine}

\section{La personne juridique}

Les droits objectifs (vie, vote, propriété) et subjectifs (vente) sont attribués uniquement aux personnes juridiques.

Les animaux ne sont pas considérés comme des personnes juridiques, mais ont le statut <<~d'être vivant doué de sensibilité~>>.

\subsection{La personne physique\label{personne-phys}}

Tout être humain est une personne physique, et est donc une personne juridique, de sa naissance\footnote{L'avortement est possible avant 12 semaines. En cas de pathologie ou de viol, il peut être autorisé jusqu'à 3 mois. Il est possible de donner des funérailles et un nom à un enfant mort-né.} à sa mort.

La mort est certaine lorsque l'arrêt complet et irréversible des organes vitaux a été constaté. Selon 1232-1 du code de la Santé Publique, \textit{<<~La mort est une absence totale de conscience et d'activité motrice spontanée avec abolition de tous les réflexes du tronc cérébral et absence de respiration spontanée.~>>} Tant qu'un décès n'est pas établi, une personne disparue est considérée vivante, donc, digne de droits.

Pour une absence, on présume que la personne est vivante: toute personne peut saisir le juge pour faire déclarer la présomption d'absence, qui s'arrête au retour ou à la preuve de la mort.

À défaut, on peut déclarer une absence (au tribunal de Grande Instance du dernier lieu de résidence) qui ne peut être déposée que 9 ans après la présomption d'absence, ou 20 ans si elle n'a pas été déclarée. Ce jugement a pour valeur d'acte de décès.

Une personne `disparue' correspond à une quasi-certitude de mort quand le décès ne peut pas être constaté par manque de cadavre.

\subsection{La personne morale\label{personne-mor}}

Les personnes morales sont des regroupements de personnes autour d'un intérêt commun. On en trouve deux catégories: publiques (l'État, les départements, les écoles, les hôpitaux...) et privées (entreprises et associations).

Une personne morale est créée lors d'un enregistrement auprès du RCS (on peut vérifier que le nom n'est pas déjà pris auprès de l'INPI), elle disparaît lors de la cessation d'activité ou la faillite.

\section{Identification des personnes}

\paragraph{Le sexe}

À la naissance de l'enfant, un acte d'état civil est établi dans lequel l'enfant est désigné comme masculin ou féminin\footnote{Les transsexuels sont des gens nés d'un sexe mais développant un caractère du sexe opposé. La Cours de Cassation a refusé les changements de sexe comme étant des cas de <<~déviance sexuelle~>>, mais a été condamnée par la Commission Européenne des Droits de l'Homme.}.

\paragraph{Le prénom et le nom de famille}

L'identité d'une personne est constituée de son prénom et de son nom de famille, elle est immuable (sauf intérêt légitime comme les consonances ou l'historique). Pour les personnes morales, on l'appelle la <<~dénomination sociale~>>, qui est choisie librement parmi celles qui ne sont pas utilisées.

Le prénom est un accessoire du nom et est choisi par les parents librement. Si l'État Civil juge que le prénom est contraire à l'intérêt de l'enfant ou d'un tiers, il peut saisir le Procureur de la République. Si le Procureur est d'accord, il peut transmettre le juge aux affaires familiales, qui lui sera compétent pour décider.

\paragraph{Le domicile}

Le domicile est un autre moyen d'identification, on ne peut en avoir qu'un (qu'on distingue du lieu de résidence ou de logement), c'est une obligation du code civil (même pour les SDF). Il détermine le tribunal, les listes électorales, le lieu de mariage... (pour les personnes morales, l'équivalent est le siège social).

\paragraph{La nationalité}

Souvent, la nationalité est le pays du domicile.

Chaque personne, physique ou morale, dispose d'un patrimoine, mais pour agir dans sa vie quotidienne on doit disposer de la capacité juridique.

\section{Capacité juridique et patrimoine\label{capacité}}

\subsection{Capacité juridique}

La capacité juridique, c'est l'aptitude a avoir des droits et des obligations -- c'est l'aptitude a exercer seul les droits dont on est titulaire.

Elle est composée de deux capacités:
\begin{itemize}
\item La capacité de jouissance est l'aptitude d'un individu à acquérir un droit.

\item La capacité de contracter est l'aptitude pour une personne d'exercer seule les prérogatives nées du droit qu'elle possède.
\end{itemize}

\subsection{Le patrimoine}

Toute personne juridique, physique ou morale dispose d'un patrimoine, on ne peut en avoir qu'un (unicité du patrimoine). Il est constitué de l'ensemble des droits et des obligations évaluées en argent.

\section{Les personnes vulnérables\label{vulnerable}}

La loi du 5 mars 2007 transforme le terme de <<~personnes incapables~>> vers <<~personnes vulnérables~>>.

Selon leurs caractéristiques, certaines personnes devront être protégées par la loi.

\subsection{Principes de vulnérabilité}

Dans l'article 435 du Code Civil sont définis les principes de vulnérabilité.

\paragraph{Principe de nécessité} Article 435 du Code Civil: \textit{<< Peuvent-être placé sous régimes de protection uniquement les personnes souffrant d'une altération de leurs facultés mentales et les personnes qui souffrent d'une altération de leurs facultés physiques. >>} Cela nécessite un certificat médical par un médecin inscrit sur la liste du Procureur de la République.

Face au refus de se présenter au rendez-vous médical, le médecin légal peut s'entretenir avec le médecin traitant et peut dresser un certificat médical de carence (depuis le 20 avril 2017).

\paragraph{Principe de subsidiarité}
Les mesures d'incapacité juridiques doivent être utilisées en dernier lieu.

\paragraph{Principe de proportionnalité}

Le juge des tutelles doit choisir un régime adapté à l'altération des facultés.

\paragraph{Principe de responsabilité}

Celui chargé d'assister la personne peut engager sa responsabilité en cas de faute dans la gestion du patrimoine ou la protection de la personne.

\subsection{Les différents actes juridiques}

\begin{itemize}
\item Actes d'administration (acheter son pain...)
\item Actes de disposition (vendre un immeuble...)
\end{itemize}

\subsection{Le cas des majeurs protégés}

\subsubsection{La sauvegarde de justice}

La sauvegarde de justice est une innovation de la loi de 1968. C'est une mesure temporaire (depuis 2017, durée maximale de 1 an renouvelable une fois) qui a pour but d'attendre le placement sous un régime plus fort (curatelle ou tutelle).

L'individu conserve sa pleine capacité juridique et aura des moyens de protection pour remettre en cause des actes juridiques qu'il aurait passé et qui lui seraient préjudiciables.

\subsubsection{La curatelle}

Dans la curatelle, le majeur va conserver une partie de sa capacité juridique, mais, pour les actes les plus graves, il sera assisté par son curateur. Le majeur peut faire seul tous les actes relatifs à la protection de la personne (actes chirurgicaux...). C'est un mécanisme qui est orienté vers l'assistance (``tout sauf'').

La curatelle est limitée dans le temps, elle dure 5 ans (et est renouvelable jusqu'à 20 ans, avant 2007 elle était illimitée).

\subsubsection{La tutelle}

À la différence de la curatelle, la tutelle est fondée sur la représentation : le tuteur va représenter le majeur pour certains actes juridiques -- on va constater l'incapacité juridique la plus étendue. 

La majeur ne pourra faire que les actes suivants:
\begin{itemize}
\item Actes usuels,
\item Actes autorisés par le juge des tutelles,
\item Actes strictement personnels.
\end{itemize}

\subsection{Les mineurs non-émancipés}

Depuis 1974, la majorité est passée à 18 ans (Article 388 du Code Civil). Depuis toujours, on affirme et continue d'affirmer que le mineur ne dispose pas d'une capacité juridique.

Il pourra faire tous les actes relatifs à sa vie juridique avec l'accord de ses représentants légaux.

Le mineur seul pourra faire seul tous les actes autorisés par l'usage, plus certains plus graves (reconnaissance d'un enfant, ouvrir un livret A à partir de 16 ans, faire son testament...)

Si un mineur effectue un acte qu'il n'aurait pas du pouvoir faire, ses représentants légaux ont le pouvoir de le déclarer nul.

\part{Organisation juridictionnelle}

Une juridiction est un organe créé par la loi afin de trancher des litiges et dont la décision a une autorité imposée aux parties.

\section{Les juridictions en ordre interne}

L'organisation judiciaire en France est dirigée par deux principes, la dualité des juridictions (ordre judiciaire et ordre administratif) et le double degré de juridiction (une affaire peut faire l'objet de deux examens).

\subsection{Les juridictions de l'ordre judiciaire}

Elles ont deux fonctions :
\begin{itemize}
\item Statuer des litiges entre particuliers (juridictions civiles)
\item Frapper de peines les auteurs d'infractions pénales (juridiction répressive).
\end{itemize}

\subsubsection{Les juridictions répressives}

Les Juridictions répressives sont chargées d'appliquer le droit pénal. Elles sont organisées selon la considération de la gravité de l'infraction.

\paragraph{Les contraventions}

Les contraventions sont les infractions les moins graves. Elles sont gérées par le \textbf{Tribunal de Police}. Elles peuvent aller jusqu'à 3000 \euro{} d'amende en cas de récidive, accompagnés de peines privatives ou restrictives comme le retrait du permis de conduire ou la saisie du véhicule.

Les contraventions sont séparées en 5 classes. La 5\ieme ~classe correspond par exemple à des violences volontaires n'ayant pas donné d'interruption de travail.

\paragraph{Les délits}

Les délits dépendent des \textbf{tribunaux correctionnels}. Ils peuvent être punis par des peines de prison allant jusqu'à 10 ou 20 ans (en cas de récidive) ou des travaux d'intérêt général.
Par exemple, il s'agit de vols, escroqueries, coups et blessures graves.

\paragraph{Les crimes}

Les crimes dépendant des \textbf{Cours d'Assise}, qui sont composées de trois juges professionnels et d'un jury de citoyens tirés au sort. Les Cours d'Assise se réunissent tous les trois mois pour des sessions de 15 jours.

\subsubsection{Les juridictions civiles}

La juridiction civile est la juridiction de droit commun. Elle a compétence pour connaître tous les litiges que la loi n'a pas attribués à d'autres juridictions.

Il existe un \textbf{Tribunal de Grande Instance} dans chaque département, qui gère uniquement les litiges supérieurs à 10 000 \euro{}.

Le \textbf{Tribunal d'Instance} a une compétence générale pour les litiges qui ne dépassent pas 10 000 \euro{}, et une compétence d'exception dans certains domaines, comme les contrats de locations, les pensions alimentaires, les crédits et la consommation.

Le \textbf{Conseil des Prud'hommes} a pour compétence de concilier ou à défaut juger les litiges relatifs aux contrats de travail entre employeur et salarié. Il n'est pas composé de magistrats mais d'élus (à parité de salariés et d'employeurs).

Le \textbf{Tribunal de Commerce} traite les affaires commerciales et les litiges relatifs aux propriétés des entreprises...

\subsection{Les juridictions de l'ordre administratif}

Les Tribunaux Administratifs, Cours Administratifs d'Appel, Conseil d'État (basé à Paris).

\section{Les juridictions européennes}

\subsection{La Cours de Justice de l'Union Européenne}

Elle est chargée de la bonne interprétation du droit communautaire pour les États-membres. Elle siège au Luxembourg et est constituée d'un juge par État-membre.

\subsection{La Cours Européenne des Droits de l'Homme}

Elle résulte de la Convention Européenne des Droits de l'Homme (adoptée en 1950). Elle siège à Strasbourg et rend des arrêts et peut prononcer des condamnations envers les États à l'origine de violations. La France est souvent condamnée pour lenteur de procédure (400 condamnations).

\section{Le procès civil}

\subsection{Compétence des juridictions}

Aptitude à connaître, instruire et juger une affaire. Il faut saisir le bon tribunal pour une action (à défaut elle sera irrecevable).

On retrouve deux règles de compétences, l'attribution (en fonction de la valeur ou de la nature du litige) et le territoire (juridiction géographiquement compétente, le tribunal du défendeur).

\subsection{Les voies de recours}

Il s'agit des manières de faire appel.

\subsubsection{Les voies de recours ordinaires}

Il s'agit de déférer une décision rendue à une juridiction supérieure (à nouveau en fait et en droit, la règle de juridiction).

La cours d'appel peut soit confirmer la décision, soit l'infirmer (la modifier).

\subsubsection{Les voies de recours extraordinaires}

Il existe une Cours de Cassation (à Paris). Elle gère les questions de droit uniquement (bonne interprétation de la règle de droit par les juges précédents).

\part{Mécanisme de responsabilités}

La responsabilité au sens général découle de la notion d'obligation, qui prennent naissance dans les actes juridiques et dans les faits juridiques.

\section{Responsabilité contractuelle}

\subsection{Formation du contrat}

Le contenu du contrat est déterminé par les parties elles-mêmes (liberté contractuelle). C'est la loi des parties. Pour être valable un contrat doit répondre à des conditions (articles 1103 et suivants du Code Civil) comme le consentement (chaque partie a émis sa volonté). Une offre est soit explicite (vente immobilière) ou tacite (vente en self en magasin).

Pour le contrat soit valable, il doit être légalement formé (selon l'article 1103 du Code Civil) :
\begin{itemize}
\item Le consentement, émission d'une volonté et acceptation du contrat à la vente, tacite ou expresse.
\item La capacité (cf \ref{capacité} page \pageref{capacité})
\item L'objet doit être possible (exister au moment du contrat) et être licite.
\end{itemize}

\subsection{Responsabilité contractuelle}

Quand le contrat est légalement formé, il doit être exécuté (on ne peut pas se soustraire aux obligations).
En cas de non-exécution de l'une des parties, on anéantie le contrat (c'est la résolution\footnote{Il ne faut pas confondre la résolution (annulation du contrat rétroactive) et la résiliation (fin du contrat prématurée).}).
On mets en place la responsabilité (réunion de plusieurs conditions, dommages, mauvaise exécution du contrat).

Il existe deux sortes d'obligations : de résultat (à atteindre) et de moyens (tout en œuvre pour remplir un engagement mais sans promettre de résultat, responsabilité engagée si négligence ou imprudence).

\section{La responsabilité délictuelle}

\begin{cquote}{Articles 1240 et 1241 du Code Civil}
Tout fait quelconque de l'homme, qui cause à autrui un dommage, oblige celui par la faute duquel il est arrivé, à le réparer.

Chacun est responsable du dommage qu'il a causé non seulement par son fait, mais encore par sa négligence ou par son imprudence.
\end{cquote}

L'article 12140 du Code Civil explique que l'on doit répondre de ses actes en dehors de tout contrat, que l'on est responsable lors des dommages (matériels, moraux) certains et directs.

La victime doit prouver le lien entre le fait générateur et les dommages.

\paragraph{Responsabilité pour faute} La faute est un acte illicite qu'une personne prudente placée dans les mêmes conditions n'auraient pas accomplies.

\paragraph{Responsabilité du fait d'autrui} On est responsable de nos faits et de ceux de ceux qui sont à notre charge (parents \& enfants, artisans \& apprentis...).

\paragraph{Responsabilité du fait des choses} Lorsqu'un dommage est causé par une chose (objet ou animal), le gardien\footnote{Le propriétaire est présumé gardien et doit prouver qu'au moment du fait dommageable, une autre personne avait la garde de la chose. L'assurance peut aider à amener la preuve.} de cette chose en est responsable et doit réparation.

\part{Principes directeurs en droit du travail}

Le droit du travail date du début du XX\ieme ~siècle. Le droit du travail est orienté vers les salariés, il vise à compenser le déséquilibre économique entre le salarié et l'employeur.

C'est l'étude des rapports juridiques qui naissent de l'exécution des missions des salariés pour le compte de l'employeur. On exclue le droit bénévole.

\section{Définition du contrat de travail}

La jurisprudence définie le contrat de travail comme une convention par laquelle une personne physique s'engage à mettre à la disposition son travail pour le compte et sous la subordination d'une autre personne moyennant rémunération.

Trois critères se dégagent de cette définition: la fourniture d'un travail, la rémunération et la subordination (autorité de l'employeur).

\section{Caractéristiques du contrat de travail}

\subsection{Les parties au contrat}

\paragraph{L'employeur} L'employeur peut être une personne physique (cf. page~\pageref{personne-phys}) ou une personne morale (cf. page~\pageref{personne-mor}).

\paragraph{Le salarié} Toute personne qui n'est pas placée en incapacité juridique (cf. page~\pageref{vulnerable}) peut conclure un contrat de travail. Si le salarié est mineur, des restrictions supplémentaires s'appliquent sur son temps de travail, ses pauses...

\subsection{Le formalisme du contrat}

Le contrat peut être écrit ou oral\footnote{un CDI peut-être non-écrit, mais l'employeur doit remettre au salarié les documents de travail : déclaration auprès de l'URSAF..., un CDD non-écrit 48 heures après le début du travail peut être requalifié en CDI par le salarié}.

Le contrat doit obligatoirement spécifier : la période d'essai\footnote{Période pendant laquelle l'employeur peut licencier sans frais et l'employé peut partir sans motif. Elle dure deux mois pour les ouvriers, trois mois pour les techniciens, quatre mois pour les cadres, et peut être renouvelée}, le salaire, l'adresse, l'horaire...

Les deux parties (l'employeur et le salarié) doivent respecter leurs engagements dans le contrat.

Le salarié doit travailler de bonne foi et ne doit pas porter atteinte à l'image de la société.

\subsection{Les différents types de contrat de travail}

Il existe différents types de contrats.

\paragraph{Contrat à Durée Indéterminée}

Il s'agit de la forme principale du contrat de travail. Il peut être rompu de façon unilatérale.

\paragraph{Contrat à Durée Déterminée}

Le CDD est un contrat d'exception possible uniquement pour l'exécution d'une tâche temporaire et précise. On ne peut recouvrir au CDD que dans des cas listés dans la loi (par exemple un accroissement temporaire de l'activité, le travail saisonnier, le remplacement...).

La période d'essai varie de 24 heures à 1 mois, dans certaines conditions il peut être renouvelé.

\paragraph{Contrat à Temps Partiel}

Un contrat à temps partiel définie un contrat dont le volume horaire est inférieur à 35 heures par semaine.

\paragraph{Contrat de l'Informaticien}

Le contrat de l'informaticien peut comporter des clauses supplémentaires comme les droits d'auteur, la confidentialité, la formation (dedit-formation)

\subsection{La fin du contrat de travail}

On trouve plusieurs manières de terminer un contrat :

\begin{itemize}
\item \textbf{Démission} avec respect du préavis
\item \textbf{Licenciement} pour motif économique ou faute professionnelle
\item \textbf{Rupture conventionnelle} : rupture amiable avec accord des deux parties
\item \textbf{Prise d'acte} : courrier recommandé à l'employeur expliquant que l'on ne souhaite plus travailler, par exemple pour des raisons de relations sociales
\item \textbf{Retraite}
\item \textbf{Décès}
\end{itemize}

\end{document}
