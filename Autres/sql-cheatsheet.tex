\documentclass[10pt,a4paper,french]{article}
\usepackage{babel}
\usepackage[utf8]{inputenc}
\usepackage[T1]{fontenc}
\usepackage{../latex/clovisai}

\makeindex
\makeglossaries

\glossaryentry{select}{SELECT}{}{Récupération de données}
\glossaryentry{from}{FROM}{}{Choisir la source}
\glossaryentry{distinct}{DISTINCT}{}{Résultats uniques sous MS SQL}
\glossaryentry{unique}{UNIQUE}{}{Résultats uniques sous Oracle}
\glossaryentry{where}{WHERE}{}{Appliquer une condition logique}
\glossaryentry{and}{AND}{}{Opérateur logique ET}
\glossaryentry{or}{OR}{}{Opérateur logique OU}
\glossaryentry{like}{LIKE}{}{Comparaison à un pattern}
\glossaryentry{in}{IN}{}{Choisir dans une liste}
\glossaryentry{between}{BETWEEN}{}{Choisir dans un intervalle}
\glossaryentry{is}{IS}{}{Est-ce qu'une valeur est nulle ?}
\glossaryentry{not}{NOT}{}{Négation}
\glossaryentry{null}{NULL}{}{Valeur nulle}
\glossaryentry{group}{GROUP BY}{}{Grouper des valeurs}
\glossaryentry{avg}{AVG}{}{Moyenne}
\glossaryentry{count}{COUNT}{}{Compter}
\glossaryentry{max}{MAX}{}{Maximum}
\glossaryentry{min}{MIN}{}{Minimum}
\glossaryentry{as}{AS}{}{Renommer une colonne ou une table (alias)}

\begin{document}

\title{SQL Cheatsheet}
\author{Ivan Canet}
\maketitle

%\begin{abstract} % ARTICLE ONLY
%Ceci est l'introduction du document.
%\end{abstract}

\tableofcontents

\section{Bases}

\subsection{\gls{select}}

\subsubsection{\gls{select} \gls{from}}

Permet de récupérer les données.
\begin{minted}{sql}
SELECT [Champs]
FROM [Table]
\end{minted}

Par exemple, pour récupérer le nom et prénom de tous les abonnés, on écrit:
\begin{minted}{sql}
SELECT Nom, Prenom
FROM Abonnes
\end{minted}

On peut aussi utiliser une astérisque pour récupérer tous les champs:
\begin{minted}{sql}
SELECT *
FROM Abonnes
\end{minted}

\subsubsection{\gls{as}}

Permet de renommer une colonne dans le contexte de la requête (la table n'est pas modifiée).
\begin{minted}{sql}
SELECT Nom AS NomDeFamille
FROM Etudiants
\end{minted}

\subsubsection{\gls{distinct} / \gls{unique}}

Permet de supprimer les doublons ; exemple: lister les noms de famille:
\begin{minted}{sql}
SELECT DISTINCT Nom
FROM Abonnes
\end{minted}

\paragraph{Microsoft vs Oracle}
Sur MS SQL on utilise `\gls{distinct}', sur Oracle on utilise `\gls{unique}' à la place.

\subsubsection{WHERE}

Permet de ne sélectionner qu'une partie de la base avec une condition logique.
\begin{minted}{sql}
WHERE [Condition]
\end{minted}

\paragraph{Opérateurs simples}
On peut utiliser `=' pour tester une égalité. Par exemple, pour avoir tous les étudiants qui s'appellent Martin:
\begin{minted}{sql}
SELECT Nom, Prenom
FROM Etudiants
WHERE Nom = 'Martin'
\end{minted}
De la même manière, on peut utiliser `!=', `$>$', `$<$', `$\leq$' et `$\geq$'.

\paragraph{Opérateurs logiques}
On peut utiliser `\gls{and}' et `\gls{or}' pour combiner les conditions:
\begin{minted}{sql}
SELECT Nom, Prenom
FROM Etudiants
WHERE Nom LIKE 'N%' OR Nom LIKE 'M%'
\end{minted}

\paragraph{Opérateur \gls{like}}
Permet de chercher pour une égalité vague; par exemple, la liste des étudiants dont le prénom commence par un `N':
\begin{minted}{sql}
SELECT Nom, Prenom
FROM Etudiants
WHERE Nom LIKE 'N%'
\end{minted}
Le \% représente ``n'importe quoi''.

\paragraph{Opérateur \gls{in}}
Permet de récupérer toutes les correspondances dans une liste. Par exemple, pour avoir les étudiants qui s'appellent `Martin', `Martine' ou `Pierre':
\begin{minted}{sql}
SELECT Nom, Prenom
FROM Etudiants
WHERE Nom IN ('Martin', 'Martine', 'Pierre')
\end{minted}

\paragraph{Opérateur \gls{between}}
Permet d'avoir les valeurs dans un intervalle, par exemple, les étudiants qui sont nés entre le 1 janvier 1998 et le 31 décembre 1998:
\begin{minted}{sql}
SELECT Nom, Prenom
FROM Etudiants
WHERE (DateNaissance BETWEEN '1998-01-01' AND '1998-12-31')
\end{minted}
Attention: certaines bases de données incluent les bornes, d'autres ne les incluent pas.

\paragraph{Opérateur \gls{is} \gls{null}}
Pour savoir si une case est vide, on peut utiliser `IS NULL':
\begin{minted}{sql}
SELECT Nom, Prenom
FROM Etudiants
WHERE Nom IS NOT NULL
\end{minted}

\paragraph{Opérateur \gls{not}}
Permet de faire la négation d'une condition. On peut l'utiliser avec `\gls{is} \gls{null}', avec `\gls{between}', avec `\gls{in}' et avec `\gls{like}'.

\subsubsection{\gls{group}}

Grouper des résultats. Exemple: dans une table qui contient toutes les notes des étudiants, grouper les notes par étudiant pour obtenir leur moyenne (\gls{avg} permet de faire la moyenne):
\begin{minted}{sql}
SELECT Nom, Prenom, AVG(Note)
FROM Notes
GROUP BY Nom, Prenom
\end{minted}
La clause du \gls{group} doit être dans le \gls{select}!

\subsubsection{HAVING}

\subsubsection{UNION / INTERSECT / EXCEPT}

\subsubsection{ORDER BY}

\subsubsection{LIMIT}

\subsubsection{OFFSET}

\subsection{Agrégations}

Les agrégations permettent de regrouper plusieurs lignes en une seule; par exemple en faisant la somme des lignes, ou alors simplement en les comptant.

Il faut toujours utiliser \gls{as}.

\subsubsection{\gls{count}}
Compter le nombre de lignes.
\begin{minted}{sql}
SELECT COUNT(*) AS Nombre
FROM Etudiants
\end{minted}

\subsubsection{\gls{avg}}
Permet de faire la moyenne d'une colonne.
\begin{minted}{sql}
SELECT AVG(Note) AS Moyenne
FROM Notes
\end{minted}

\subsubsection{\gls{min} et \gls{max}}
Trouver le minimum ou le maximum d'une colonne. Exemple: trouver la meilleure note:
\begin{minted}{sql}
SELECT MAX(Note) AS Note
FROM Notes
\end{minted}

\section{}

\section{Transactions}

\appendix % Annexes, ARTICLE & BOOK

%\bibliography{•} % THE .BIB FILE HERE, WITHOUT THE EXTENSION
%\cprintindex
\printglossaries

\end{document}
