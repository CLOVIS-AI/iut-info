\documentclass[10pt,a4paper,french]{article}
\usepackage{babel}
\usepackage[utf8]{inputenc}
\usepackage[T1]{fontenc}
\usepackage{../latex/clovisai}

\begin{document}

\newminted[exalgo]{text}{mathescape,escapeinside=||}

\title{Ex-Algo: Syntax Cheatsheet}
\author{Ivan Canet}
\maketitle

\begin{abstract}
Ce document résume toute la syntaxe connue d'Ex-Algo, avec des exemples. Le but est de regrouper ce qui est connu en un seul document facile à parcourir.

Ce document n'est pas un cours d'ex-algo : juste une liste de toutes les syntaxes.
\end{abstract}
\tableofcontents

\section{Syntaxes élémentaires}

\subsection{Fonctions et Actions}

En Ex-Algo, nous écrirons toujours le code dans une Fonction ou une Action.

Comme en maths, une Fonction peut prendre des paramètres et donne un résultat. Une Action et peut aussi prendre des paramètres, mais elle ne peut pas donner de résultat.

\subsubsection{Déclaration et appel}

\paragraph{Déclaration d'une Action}
\begin{exalgo}
Action uneAction(PARAMÈTRES)
Début
	# ... le code de l Action
Fin
\end{exalgo}

\paragraph{Exemple d'Action}
\begin{exalgo}
Action bonjour()
Début
	Écrire "Hello World"
Fin
\end{exalgo}

\paragraph{Déclaration d'une Fonction}
\begin{exalgo}
Fonction uneFonction(PARAMÈTRES)
Début
	# ... le code de la Fonction, avec un Retourner
Fin
\end{exalgo}

\paragraph{Exemple de Fonction}
$carré: x \rightarrow x^2$.
\begin{exalgo}
Fonction carré(E x: Réel): Réel
Début
	Retourner x^2
Fin
\end{exalgo}

\subsubsection{Paramètres}

\paragraph{Mode}
Les paramètres ont un ``mode'':
\begin{description}
\item[Entrée (E)] La Fonction/Action récupère la valeur du paramètre. Elle peut le modifier, mais les modifications ne seront pas visibles après qu'elle se termine.
\item[Sortie (S)] La Fonction/Action va modifier le paramètre, et sa nouvelle valeur sera accessible à la fin de l'appel. Il est interdit d'initialiser la variable avant le début de la fonction (elle doit être vide lors de l'appel)
\item[Entrée-Sortie (ES)] La Fonction/Action récupère la valeur du paramètre. Elle peut le modifier, et les modifications existeront toujours après qu'elle se termine.
\end{description}

\paragraph{Syntaxe des paramètres}
\begin{exalgo}
...(MODE NOM: TYPE, MODE NOM: TYPE,...)...
\end{exalgo}

\paragraph{Exemple de paramètres}
Cet exemple n'est rien de plus, ce code devrait être une Fonction et retourner le résultat pour être plus propre.
\begin{exalgo}
Action max(E premier: Réel,
           E second: Réel,
           S résultat: Réel)
Début
    Si premier |$\geq$| second
        résultat |$\leftarrow$| premier
    Sinon
        résultat |$\leftarrow$| second
Fin
\end{exalgo}

\subsubsection{Variables}

Les variables permettent de stocker des différentes données.

\paragraph{Déclaration de variables}
Les variables sont déclarées au début d'une fonction/action:
\begin{exalgo}
Action test()
Variables
    i: Entier
Début
    Pour i de 1 à n Faire
        Écrire i
Fin
\end{exalgo}

\paragraph{Constantes}
Les constantes sont déclarées comme les variables, mais peuvent être assignées sur place:
\begin{exalgo}
Action test()
Constantes
    t: ChaîneDeCaractères |$\leftarrow$| "Tour: "
Variables
    i: Entier
Début
    Pour i de 1 à n Faire
        Écrire t, i
Fin
\end{exalgo}

\subsubsection{Types}

\paragraph{Liste des types primitifs}
\begin{description}
\item[Réel -- $\mathbb{R}$]
\item[Entier -- $\mathbb{N}$]
\item[Caractère]
\item[ChaîneDeCaractères]
\end{description}

\subsection{Conditionnelles}

\subsubsection{Si}

\paragraph{Syntaxe}
\begin{exalgo}
Si CONDITION Alors
    CODE
SinonSi CONDITION Alors
    CODE
Sinon
    CODE
\end{exalgo}

\paragraph{Rien}
Si un morceau de code est vide, on peut utiliser Rien:
\begin{exalgo}
Si CONDITION Alors
    Rien
\end{exalgo}

\subsubsection{Selon que}

\paragraph{Utilisation générale}
\begin{exalgo}
Selon Que
    CONDITION
        CODE
    CONDITION
        CODE
    Sinon
        CODE
\end{exalgo}

\paragraph{Utilisation pour dénombrement d'une variable}
\begin{exalgo}
Selon que n vaut
    VALEUR
        CODE
    VALEUR
        CODE
    Sinon
        CODE
\end{exalgo}

\subsection{Boucles}

\subsection{Divers}

\section{Structures de données}

\subsection{Tableaux}

\section{•}

\end{document}
