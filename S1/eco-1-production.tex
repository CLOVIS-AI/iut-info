\documentclass[10pt,a4paper,french]{article}
\usepackage{babel}
\usepackage[utf8]{inputenc}
\usepackage[T1]{fontenc}
\usepackage{../latex/clovisai}

\makeindex

\begin{document}

\title{Économie - Production}
\author{Ivan Canet}
\maketitle

%\begin{abstract} % ARTICLE ONLY
%Ceci est l'introduction du document.
%\end{abstract}

\tableofcontents

\part{Définition et mesure de la production}

\section{Définition de la production}

\paragraph{Production}

La production est le fait de transformer du travail et du capital en biens et services.

Les sociétés financières, publiques, privées, les ménages, les associations, les partis politiques, sont autant
d’organismes qui produisent.

\paragraph{Production marchande\mainentry{Production!Marchande}}

La production marchande désigne le cas où le but de la production est d’obtenir un chiffre d’affaire\entry{Chiffre d'affaire} ;

\[
\textbf{Bénéfice} = \text{Chiffre d'affaire} - \text{Coûts de production}
\]

Le chiffre d’affaire\mainentry{Chiffre d'affaire} est constitué du prix de vente multiplié par la quantité, les coûts de production
désignent le travail et le capital.

\paragraph{Production non-marchande\mainentry{Production!Non-Marchande}}

Dans le cas d’une production non-marchande, l’entreprise ou l’organisation à l’origine de la production n’a
pas comme but le bénéfice, mais plutôt l’accès public à un bien ou un service, par exemple.

\section{Mesures de la production}

\subsection{Valeur ajoutée\mainentry{Valeur ajoutée}}

La Valeur Ajoutée constitue un indice des richesses créées par une entreprise ;

\[
\textbf{Valeur ajoutée} = \text{Chiffre d'affaire} - \text{Consommations intermédiaires}
\]

\subsection{Basées sur le PIB}

\subsubsection{Produit Intérieur Brut}

\paragraph{PIB en valeur\mainentry{PIB!en valeur}}

Le PIB en valeur (ou PIB nominal\mainentry{PIB!nominal}) est un indice de la production d’un pays.

\[
\text{PIB}_\text{Valeur} = \sum (\text{VA}+\text{TVA})+\text{Douanes}
\]

\paragraph{PIB en volume\mainentry{PIB!en volume}}

Dans le cas du PIB en valeur, on ne peut pas comparer plusieurs années ou plusieurs pays ensemble, à
cause des différences de valeur des monnaies géographiquement et au cours du temps. Pour comparer les
productions, on utilise donc le PIB en volume (ou PIB réel\mainentry{PIB!réel}) ;

\[
\text{PIB}_\text{Volume} = \frac{\text{PIB}_\text{Valeur}}{\text{IPC}}
\]

où l’IPC est un indice de l’inflation calculé tous les ans par l’INSEE\footnote{Institut National des Statistiques et des Études de l'Économie} pour permettre la déflation.

Les pays ayant le plus grand PIB en volume\entry{PIB!en volume} au monde sont les États-Unis, la Chine, le Japon, l’Allemagne, le Royaume-Uni et la France.

\paragraph{PIB par habitant\mainentry{PIB!par habitant}}
Le PIB par habitant correspond au PIB en volume divisé par le nombre d’habitants.

Les pays ayant le plus grand PIB par habitant sont le Luxembourg, la Suisse, la Norvège, l’Irlande, Macao
et le Qatar.

\paragraph{PIB en PPA\mainentry{PIB!en PPA}}

Le PIB en PPA désigne une conversion du PIB vers une monnaie unique, pour permettre les
comparaisons du PIB entre les différents États.

\subsubsection{Revenu National Brut}

Le RNB\mainentry{RNB} est un indice basé sur le PIB\entry{PIB}, auquel on ajoute les facteurs de travail et de capitaux extérieures,
auxquels on soustrait les facteurs de travail et de capitaux intérieurs ;

\[
\text{RNB} = \text{PIB} + \text{exports} - \text{imports}
\]

\subsubsection{Limites du PIB}

Le PIB a certaines limites ;

\begin{itemize}
\item Le PIB ne prend pas en compte l’économie informelle ou souterraine (production domestique, fausses
déclarations, marchés illicites),
\item Les inégalités entre membres de la société ne sont pas visibles non plus,
\item Les dégâts causés au patrimoine sont ignorés aussi.
\end{itemize}

\subsection{Autres indicateurs de production}

\subsubsection{Baromètre des Inégalités et de la Pauvreté}

Le BIP 40\mainentry{BIP 40} est un autre indice qui prend en compte 58 facteurs (40 originalement, d’où le nom) comme
l’emploi, le revenu, la santé, l’éducation, le logement et l’injustice.

\part{Définition et mesure de la croissance}

\section{Définition}

\begin{cquote}{François Perroux}
La croissance économique est l’augmentation soutenue pendant une période longue d’un indicateur de production.
\end{cquote}

Mathématiquement, la croissance\mainentry{Croissance} économique est donc ;

\[
\text{Croissance} = \frac{\text{arrivée} - \text{départ}} {\text{départ}} \times 100
\]

\part{Définition et mesure du développement}

\paragraph{Historique}

Dans les années 40, à l’époque de la colonisation, les pays les moins avancés sont appelés Pays Sous-Développés
; mais dans les années 2000, on les renomme Pays Moins Avancés.

\section{Définition}

\begin{cquote}{François Perroux, Études, janvier 1961}
Le développement est la combinaison des changements mentaux et sociaux d’une population qui la rendent apte à faire croître, cumulativement et durablement, un produit réel global.
\end{cquote}

On voit donc que la mesure du développement est quantitative (comme le PIB), mais aussi qualitative (sur l’aspect social et culturel).

On remarque aussi que le développement rend la croissance\entry{Croissance} irréversible.

\section{Indice de Développement Humain}

\paragraph{Définition}

L’Indice de Développement Humain (IDH\mainentry{IDH}) est un indice du développement d’un pays. Il est compris entre 0 (niveau de vie horrible) et 1 (niveau de vie idéal).

Il se calcule en faisant la moyenne entre l’espérance de vie à la naissance, le niveau d’instruction (longueur des études et niveau attendu), ainsi que le RNB ou le PIB/hab.

On considère que les pays développés ont un IDH supérieur à 0,8. Un IDH moyen est compris entre 0,5 et 0,7, et un IDH faible est inférieur à 0,5.

\paragraph{Pays ayant le plus grand IDH}

Les pays ayant l’IDH\entry{IDH} le plus élevé sont la Norvège, l’Australie, le Danemark, Singapour, l’Irlande,
l’Islande, le Canada, les USA et la France.

\part{Conclusion}

\paragraph{La développement durable}

Épuisement des ressources naturelles
La croissance\entry{Croissance} actuelle épuise les ressources non renouvelables en matière première et en énergie, et
rejette en quantité grandissantes des déchets qu’elle ne sait pas traiter. Ce ne sont pas seulement les
écologistes qui le disent ; tous les experts soulignent les dangers que nous faisons courir à notre planète en
maintenant notre modèle de croissance.

\paragraph{Inégalités de richesses}

La croissance actuelle, parce qu’elle est très inégalement répartie, exacerbe les tensions entre les pays. Le
risque de conflits majeurs n’est pas à écarter si le fossé qui sépare les pays développés des autres ne tend pas
à se résorber, ce qui n’est pas le cas aujourd’hui.

\paragraph{Comment construire une croissance soutenable ?}

Une croissance\entry{Croissance} est « soutenable » si elle est acceptable par tous à court et long terme, c’est-à-dire qu’elle ne met pas en danger la croissance future. On l’appelle aussi développement durable.

\begin{cquote}{Rapport Brundtland, Notre avenir à tous, 1987}
Il s’agit d’un développement qui satisfait les besoins de chaque génération, à
commencer par ceux les plus démunis, sans compromettre la capacité des générations
futures à satisfaire les leurs.
\end{cquote}

\appendix % Annexes, ARTICLE & BOOK
\printindex

\end{document}
