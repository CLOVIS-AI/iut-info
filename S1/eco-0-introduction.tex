\documentclass[10pt,a4paper,french]{article}
\usepackage{babel}
\usepackage[utf8]{inputenc}
\usepackage[T1]{fontenc}
\usepackage{../latex/clovisai}

\makeindex

\begin{document}

\title{Économie -- Introduction}
\author{Ivan Canet}
\maketitle

\begin{abstract} % ARTICLE ONLY
Introduction au cours d'économie, définissant les sciences économiques et présentant les principaux courants de pensée économique, ainsi que Keynes, Marx ou Smith.
\end{abstract}

\tableofcontents

\section{L'objet de la science économique}

\subsection{Bases de l'économie}

\paragraph{Besoins illimités}
Les besoins\entry{Besoin} des humains ne peuvent, par nature, être satisfaits, et restent toujours illimités.

\paragraph{Biens et services illimités}
Contrairement aux besoins, les biens\entry{Bien} et les services\entry{Service} sont toujours produits en quantités limitées (sauf l'air).

\subsection{Définition}

Les principaux facteurs de production sont le travail\entry{Travail} et le capital\entry{Capital}.
\begin{cquote}{}
L'économie se définie comme l'étude de la satisfaction des besoins humains illimités avec des biens limités. On étudie les biens et les richesses, leur répartition, la production, les échanges.
\end{cquote}

\section{Méthode d'analyse économique}

On sépare l'analyse de l'économie en deux parties:
\begin{description}
\item[Micro-économie\entry{Économie!Micro}] On s'intéresse aux agents économiques.
\item[Macro-économie\entry{Économie!Macro}] On s'intéresse à leurs interactions entre eux.
\end{description}

\section{Principaux courants de pensée économique}

\subsection{Le courant libéral\mainentry{Courant!Libéral}}

\subsubsection{Mouvement classique\mainentry{Courant!Libéral!Classique}}

Le courant libéral a été créé par\entry{Smith}Adam Smith, dans son livre, ``\textit{Recherche sur la nature et les causes de la richesse des nations}''.

\subsubsection{Mouvement néo-classique\mainentry{Courant!Libéral!Néo-classique}}

La théorie de l’avantage\entry{Avantage!absolu}absolu, puis de l’avantage\entry{Avantage!comparatif}comparatif, explique que tout État doit se spécialiser dans un domaine dans lequel il est le meilleur – ou au moins le moins mauvais – pour permettre à tous les États participant au marché d’atteindre un optimum commercial.

\subsubsection{L'offre\entry{Offre} et la demande\entry{Demande}}

Jean-Baptiste Say\entry{Say}théorise que l'offre crée sa propre demande, sur tous les marchés.

\subsection{Courant Marxiste\mainentry{Courant!Marxiste}}

Karl Marx\entry{Marx}est un historien et psychologue, qui s’intéresse aux rouages du capitalisme et leur critique. Il se fait exiler d’Allemagne, de Prusse puis de France, et fini sa vie en Angleterre, à Londres.

\subsubsection{Théorie globale}

\paragraph{Classes sociales\entry{Classes sociales}}
D’après la philosophie de\entry{Marx}Marx, il existe deux catégories sociales, appelées `classes sociales'. Cette séparation s’appelle le matérialisme\mainentry{Matérialisme Social}social.
\begin{description}
\item[Capitalisme\entry{Capitalisme}] Ceux qui possèdent les moyens de production,
\item[Prolétariat\entry{Prolétariat}] Ceux qui possèdent la force de travail.
\end{description}

\paragraph{Théorie de l'économie}
Selon\entry{Marx}Marx, le capitalisme amène à une situation dans laquelle on a: \[ \text{Valeur du travail} > \text{Coût du travail} \] d'où on obtient un surplus, appelé la\entry{Plus-value}\textbf{plus-value}.

Cette plus-value, ou\entry{Profit}profit, correspond à une exploitation du prolétariat. Pour\entry{Marx}Marx, les classes sociales\entry{Classes sociales}vont ``prendre conscience d'intérêts communs à agir'', impliquant une lutte des\entry{Classes sociales!Lutte des}classes, qui va conduire au socialisme, puis au communisme.

\subsection{Courant Keynésien\mainentry{Courant!Keynésien}}

Keynes\entry{Keynes}est né à Cambridge de parents universitaires, s’intéresse beaucoup aux mathématiques, puis à l’économie.

\subsubsection{Éléments théoriques}

\paragraph{L'offre\entry{Offre} selon la demande\entry{Demande}}
Selon lui, la demande est influencée par l'offre, et non l'inverse:
\[ \text{Demande} = \text{Consommation des ménages} + \text{Consommation des entreprises} + \text{Investissements} \]

Ainsi, pour obtenir une augmentation de la demande effective, il faudrait une augmentation de la production, des embauches et des revenus.

\paragraph{Interventions de l'État\entry{État!Rôle de}}
Pour Keynes\entry{Keynes}, l'État n'intervient qu'en cas de crises, et ne doit jamais intervenir en temps normal.

\section{Circuit économique : interactions entre agents}

Pour Keynes\entry{Keynes}, l'économie se résume aux interactions entre les différents agents.

Les différents agents sont:
\begin{description}
\item[Les entreprises] produisent des biens,
\item[Les banques] financent l'économie à travers les crédit,
\item[Le public] propose des services non-marchands\entry{Production!Non-Marchande},
\item[Les ménages] consomment et
\item[Le reste du monde] importe et exporte.
\end{description}

\appendix % Annexes, ARTICLE & BOOK
\printindex

\end{document}
