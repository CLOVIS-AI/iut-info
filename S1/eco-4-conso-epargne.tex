\documentclass[10pt,a4paper,french]{article}
\usepackage{babel}
\usepackage[utf8]{inputenc}
\usepackage[T1]{fontenc}
\usepackage{../latex/clovisai}

\makeindex
\makeglossaries

\begin{document}

\title{Économie -- Consommation \& Épargne}
\author{Ivan Canet \& Guillaume Ruffin}
\maketitle

%\begin{abstract} % ARTICLE ONLY
%Ceci est l'introduction du document.
%\end{abstract}

\tableofcontents

\section{Consommation}

\subsection{Définition}

\begin{cquote}{Consommation\mainentry{Consommation}}
Action d'utiliser ou de détruire immédiatement ou progressivement un bien ou un service.~
\end{cquote}

Il existe plusieurs types de consommation:
\begin{itemize}
\item Collective\entry{Consommation!Collective} ou\entry{Consommation!Individuelle} individuelle,
\item Marchande\mainentry{Consommation!Marchande} (achetée à un prix) ou non-marchande\mainentry{Consommation!Non-marchande} (gratuite ou quasi-gratuite: prix inférieur au coût de production),
\item Finale\mainentry{Consommation!Finale} (satisfaire directement les besoins, principalement ceux des ménages) ou intermédiaire\mainentry{Consommation!Intermédiaire} (consommée pendant la production).
\end{itemize}

Depuis le SEC 95\entry{SEC 95} (Système Européen de la Comptabilité, 1996), la comptabilité nationale calcule deux agrégats pour la consommation des ménages : la dépense de consommation (dépenses directement supportées par les ménages ; 1186,1 milliards d’euro\footnote{D'après les comptes nationaux de\entry{INSEE} l'INSEE, 2015}) et la consommation effective (dépenses de consommation et part socialisée de la consommation : éducation, santé ; 1576,5 milliards d'euro\footnote{D'après les comptes nationaux de\entry{INSEE} l'INSEE, 2015}).

\subsection{Loi d'Engels}

Les économistes utilisent ces données pour calculer des coefficients budgétaires, qui représentent la part de la consommation assignée à chaque poste (par exemple l'alimentation\ldots). Le calcul se fait grâce à ;
\[ \text{Coefficient budgétaire} = \frac{\text{Dépenses du poste}}{\text{Total des dépenses}} \times 100 \]

On voit ainsi que les secteurs de l'alimentation, de l'habillement, des meubles ont fortement baissé, mais
les transports, la santé et le logement augmentent.

On peut utiliser ces données, appelées Loi\mainentry{Loi d'Engels} d'Engels, pour mesurer l'avancement d'un pays:
\begin{itemize}
\item Dans un pays pauvre, les coefficients budgétaires de l'alimentation et de l'habillement sont élevés
\item Dans un pays riche, le logement est plus proéminent
\end{itemize}

\subsection{Déterminants de la consommation}

On peut distinguer deux catégories de déterminants économiques comme le revenu et le sociologique.

\subsubsection{Revenu}

\paragraph{Approche Keynésienne: revenu\mainentry{Revenu!Courant} courant}

\begin{cquote}{J.M.\entry{Keynes} Keynes}
En moyenne, la plupart du temps, les hommes pendent à accroître leur consommation à mesure que leur revenu croît, mais non d'une quantité aussi grande que l'accroissement du revenu.
\end{cquote}

\[ \text{Propension moyenne à consommer} \Leftrightarrow \frac{\Delta \, \text{Consommation}}{\Delta \, \text{Revenu}} < 1 \]
La propension moyenne à consommer à tendance à diminuer sur le long-terme; elle est plus faible que pour les pauvres.

Les libéraux critiquent cette théorie du revenu courant en annonçant que la fonction de consommation est instable dans le temps et impropre à toute prévision.

\paragraph{Approche libérale: revenu\mainentry{Revenu!Permanent} permanent}
La théorie du revenu permanent a été fondée par Milton Friedman\entry{Friedman} (1912 – Nov. 2006), le chef de file de l’école monétariste et prix Nobel en 1976.

Le revenu permanent est le revenu qu'un individu considère comme normal ; donc la moyenne pondérée des revenus courants constatés : ``La somme qu’un consommateur peut consommer en maintenant constante la
valeur de son capital.''

Cette théorie implique que l'individu est rationnel.

\subsubsection{Déterminants non-économiques}

On retrouve plusieurs critères non-économiques, parmi lesquels:
\begin{itemize}
\item La profession et la catégorie socio-professionnelle (PCS),
\item L'âge: un individu âgé consomme plus de services de santé qu'un adolescent,
\item Le mode de vie: la consommation est en partie influencée par le mode de vie de l'individu,
\item Effet d'imitation: la consommation répond parfois au besoin de copier la consommation de la classe sociale supérieure.
\end{itemize}

\paragraph{L'engagement de Pierre Bourdieu}
Pierre Bourdieu\entry{Bourdieu} (1\ier~Juin 1930 -- 23 Janvier 2002), issu d'une famille agricole dans un milieu défavorisé, fait de longues études en sociologie et devient, par son engagement public, un des acteurs principaux de la vie intellectuelle française. Il critique la jupe comme «~corset invisible~» et la reproduction sociale causée par l'école.

Son œuvre sociologique est dominée par une analyse des mécanismes de reproduction des hiérarchies sociales, faisant une place très importante aux facteurs culturels et symboliques.

\subsection{Une démarche sociale}

\begin{cquote}{Jacques Séguéla à propos de Nicolas Sarkozy, 2009}
Comment peut-on reprocher à un Président d'avoir une Rolex? Une Rolex, enfin, tout le monde a une Rolex\ldots Si à 50 ans on n'a pas une Rolex, on a quand même raté sa vie.~
\end{cquote}

\subsubsection{Consommation ostentatoire}

La consommation ostentatoire est une orientation destinée ; elle sert à montrer un statut social, un mode de vie ou une personnalité, ou à faire croire aux autres que l'on possède ce statut social, ce mode de vie ou cette personnalité.

\subsubsection{Types de revenus}

\paragraph{Revenus primaires}
Les revenus primaires sont les revenus du travail ou les revenus d'activité. Ils sont mixtes : ils rémunèrent le travail et le capital, mais aussi les propriétés (dividendes, intérêts, loyers perçus).

\paragraph{Revenus de transfert}
Ils sont versés quand le ménage remplit un certain nombre de conditions, ce sont des revenus sociaux (allocations familiales, RMI, RSA, retraites\ldots).

\paragraph{Revenu disponible}
Il s'agit de la somme du revenu primaire et des revenus de transfert moins les allocations sociales et les impôts directs. En 2014, le revenu disponible brut annuel moyen par ménage est de 36~000\euro.

\subsection{Pauvreté}

\begin{cquote}{Pauvreté}
Un individu ou un ménage est considéré comme pauvre lorsqu'il vit dans un ménage dont le niveau de vie est inférieur au seuil de pauvreté.
\end{cquote}

L'INSEE\entry{INSEE} et l'Eurostat\entry{Eurostat} mesurent la pauvreté monétaire d'une manière relative, alors que d'autres pays (États-Unis, Canada) ont une approche absolue. Dans l'approche en termes relatifs, le seuil peut-être égal à 50\% du niveau de vie médian ; Eurostat privilégie un seul à 60\% du niveau de vie médian.

Depuis 1970, on constate une augmentation des revenus des pauvres.

\subsection{En Aquitaine}

En milieu urbain, on trouve des inégalités de revenu plus marquées : un Aquitain sur huit est sous le seuil de pauvreté ; la pauvreté est plus fréquente en milieu rural qu'en milieu urbain.

\section{Épargne}

\subsection{Définition}

L'épargne est la fraction du revenu disponible qui n'est pas consommée.

En fournissant des capitaux nécessaires aux entreprises pour les investissements, elle contribue à la croissance et à la modernisation de l'outil productif.

Aujourd'hui, le taux d'épargne vaut environ 14\% du revenu disponible en France. Dans les années~80, il a chuté en raison des politiques de rigueur : moins de prêts, donc un taux d'intérêt plus élevé. Depuis les
années~90, avec la baisse du taux d'inflation, le pouvoir d'achat progresse.

\subsection{Motifs et forme}

On retrouve trois motifs de l'épargne: le patrimoine, la précaution, la spéculation.

L'épargne peut être soit liquide (transformable très rapidement en monnaie: livret A, livret Jeune\ldots On parle de thésaurisation\footnote{Mettre l'argent dans une zone sans taux d'intérêt}\mainentry{Thésaurisation}) ou investie (dans des placements, comptes, plans, titres: compte d'épargne, actions, obligations\ldots).

Le taux d'épargne augmente, ce qui peut être expliqué par une baisse du taux d'inflation, la baisse du pouvoir d'achat et l'inquiétude des ménages face au financement de la sécurité sociale et des retraites.

\subsection{Épargne et investissement}

\paragraph{Point de vue\entry{Néoclassique} néoclassique}
Pour les économistes néoclassiques, le niveau d'épargne est déterminé par le taux d'intérêt. En effet, dans
le cadre de la théorie néoclassique, l'agent économique cherche à maximiser son utilité ; lorsqu'il est amené à
faire un arbitrage entre consommation et épargne, il va considérer ce que lui rapportera l'épargne ;
autrement dit, il va considérer le taux d'intérêt.

Si celui-ci est élevé, l'agent sera incité à épargner puisque épargner permettra d'assurer des revenus
importants dans le futur.

À l'inverse, lorsque le taux d'intérêt est faible, l'agent a tendance à peu épargner, car l’épargne ne lui
rapportera que peu de revenus dans le futur.

Pour les néoclassiques, c'est donc l’épargne qui précède la consommation.

\paragraph{Point de vue\entry{Keynes} Keynésien}
L'approche Keynésienne du comportement d'épargne est tout autre : c'est ici la consommation qui précède
l'épargne. Le niveau d'épargne n'est pas déterminé par le taux d'intérêt mais par le niveau de revenu de
l'agent. Celui-consomme d'abord et attribue le reste de son revenu (celui qui n'a pas été consommé) à
l'épargne ou à la thésaurisation en fonction du taux d'intérêt.

Si le taux d'intérêt est élevé, alors l'individu aura une préférence pour l'épargne. Par contre si le taux
d'intérêt est faible, il penchera en faveur de thésaurisation.

Là est la grande différence avec les néoclassiques car Keynes prend en compte le caractère irrationnel des
agents économiques avec la thésaurisation.

\section*{Conclusion}
Le comportement d'épargne n'est pas neutre quand à l'économie appréhendée globalement. En effet, une insuffisance d'épargne peut porter préjudice à l'investissement et donc à l'activité économique dans le futur. À l'inverse, un excès d'épargne peut être préjudiciable à la demande, et donc, là encore, l'activité économique.

\appendix % Annexes, ARTICLE & BOOK

%\bibliography{•} % THE .BIB FILE HERE, WITHOUT THE EXTENSION
\printindex
%\cprintglossaries

\end{document}
