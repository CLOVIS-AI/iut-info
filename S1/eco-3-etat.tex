\documentclass[10pt,a4paper,french]{article}
\usepackage{babel}
\usepackage[utf8]{inputenc}
\usepackage[T1]{fontenc}
\usepackage{../latex/clovisai}

\makeindex
\makeglossaries

\newacronym{bce}{BCE}{Banque Centrale Européenne}
\newacronym{pib}{PIB}{Produit Intérieur Brut}

\begin{document}

\title{Économie -- Le rôle de l'État}
\author{Ivan Canet \& Guillaume Ruffin}
\maketitle

\begin{abstract} % ARTICLE ONLY
Quel est le rôle de l'État ? Quels sont les différences entre un État-gendarme, un État-providence ? Quels pouvoirs lui sont attribués, et comment est calculé son budget ?
\end{abstract}

\tableofcontents

\section{Introduction}

\paragraph{État gendarme}
Un État gendarme se réduit aux fonctions fondamentales : la sécurité intérieure (la Police), la sécurité extérieure (l’Armée) et la Justice. L’État peut aussi s’occuper des infrastructures de transport, pour permettre de ne pas entraver les sécurités.

\paragraph{État providence}
L’État providence a un rôle de Welfare-State ; c’est l’idée de Keynes, l’État intervient dans les domaines économiques et sociaux pour agir sur le bien-être de la population.

\paragraph{Les pouvoirs de l'État}
L’État a trois pouvoirs ; le pouvoir législatif (exercé par le parlement : l’Assemblée Nationale et le Sénat), le pouvoir exécutif (exercé par le président) et le pouvoir judiciaire (la Justice).
Proches des citoyens se trouvent les collectivités, les régions (contrôlées par le président de la région), les départements (par le conseil général), et les communes (Maire + conseil municipal).

L’État fait aussi référence à la Sécurité Sociale.

\section{Le poids de l'État}

\subsection{Budget}
Le budget est l’ensemble des comptes qui dérivent pour une année civile toutes les ressources et toutes
les charges de l’État.

Les projets de loi de finances et de loi de financement de la sécurité sociale sont soumis en premier lieu à
l’Assemblée Nationale.

\subsubsection{Les recettes}
Le total des recettes fiscales nettes est de 243,842 milliards d’euros.
\begin{description}
\item[Les recettes fiscales nettes] sont constituées des impôts directs et indirects desquels on a déduit les remboursements et dégrèvements (90 \%).
\item[Les recettes non fiscales]comprennent les autres recettes de l’État (exploitations industrielles et
commerciales et établissements publics à caractère financier, produits et revenus du domaine de
l’état, taxes, redevances et recettes assimilées, etc. (10 \%)
\item[Les prélèvements sur les recettes de l'État] sont constitués des recettes que l’État affecte aux collectivités locales et du prélèvement réalisé au profit de l’Union Européenne. Total des prélèvements 2017 : 63,064 milliards d’euros.
\end{description}

\subsubsection{Les dépenses}
On distingue les dépenses par mission et par nature.

\paragraph{Dépenses par mission}
Les dépenses de l'État sont présentées par mission, ce qui permet de mieux appréhender à quelle politique publique (ou mission) sont affectées les ressources.

\begin{table}[h]
\centering
\begin{tabular}{r|cc}
Postes des dépenses les plus importantes & Part du budget & Milliards d'\euro \\
\hline
Enseignement scolaire & 16.1 \% & 71.6 \\
Défense & 10.6 \% & 47.2 \\
Dette & 9.4 \% & 41.6 \\
Recherche de l'Enseignement Supérieur & 6.2 \% & 27.6
\end{tabular}
\end{table}

\paragraph{Dépenses par nature}
Les dépenses par nature présentent les dépenses groupées par type.
\begin{enumerate}
\item Personnels (Fonctionnaires)
\item Interventions
\item Fonctionnement (Entretien des bâtiments\ldots)
\item Charge de la Dette
\item Investissements (Routes\ldots)
\end{enumerate}

\subsection{Prélèvements obligatoires}
Les prélèvements obligatoires sont les impôts et cotisations sociales reçus par les administrations publiques (les administrations publiques centrales, locales et de sécurité sociale) et les institutions européenne.

Ils représentent 44 \% du \gls{pib} en 2016, et environ 40 \% sur les 10 dernières années.

\subsection{Définit budgétaire}
Le déficit budgétaire est la situation dans laquelle les recettes de l’État (hors remboursement d’emprunt) sont inférieures à ses dépenses (hors emprunt) au cours d’une année. C’est donc un solde négatif.

Le déficit public correspond au déficit de toutes les administrations publiques (État, administrations de la sécurité sociale, administrations locales et organismes divers d’administration centrale).

\subsection{Dette publique}
La dette publique correspond à l’ensemble des emprunts publics, c’est-à-dire contractés par des administrations publiques (État, collectivités locales, sécurité sociale). L’accumulation des déficits au cours
des années constitue la dette publique.

De 2007 à 2014, la dette augmentait ; elle stagne depuis. En 2016, elle atteignait 97,6 \% du \gls{pib}.

\section{Fonctions de l'État providence}

Selon R. Musgrave (1959), l'État s'occupe de :
\begin{description}
\item[Allocation des ressources] lutte contre les externalités négatives (pollution des entreprises),
\item[Redistribution des revenus] prélèvements et affectations,
\item[Stabilisation de la conjoncture] croissance équilibrée et lutte contre l'inflation et le chômage par la politique conjoncturelle. On parle aussi de fonction de régulation.
\end{description}

L'État utilise la politique économique pour satisfaire ses fonctions. On distingue la politique structurelle et conjoncturelle.

\subsection{Politique structurelle}

La politique structurelle est une politique de long terme qui vise à modifier en profondeur les structures de l'économie ; par exemple les 35 heures, l'éducation ou la réforme des retraites.

\subsection{Politique conjoncturelle}

Les quatre objectifs principaux de la politique conjoncturelle sont:
\begin{itemize}
\item Un taux de chômage minimum,
\item Un taux de croissance élevé,
\item Une inflation minime,
\item Un équilibre de la balance extérieure.
\end{itemize}

La réussite de ces quatre objectifs est schématisée par le carré magique de Kaldor.

\subsubsection{Politique monétaire}
Dans le cadre de l’UEM ou de la zone Euro, les États-membres ne peuvent mener qu’une politique budgétaire.

La \gls{bce} décide de diminuer le taux d’intérêt directeur ($\approx$ 1 \%), ce qui permet de stimuler les crédits donc la consommation.

\subsubsection{Politique budgétaire}
Les pays de l’Union Européenne pratiquent une politique budgétaire de rigueur : les Critères du Pacte de Stabilité de l’Union spécifie qu’un État-membre ne peut avoir un déficit public de plus de 3 \% de son \gls{pib} et la dette ne peut être supérieure à 60 \% de son \gls{pib}.

Cette politique a comme conséquence une diminution des dépenses de l’État et une augmentation des impôts. Or, une baisse des dépenses de l’État peut diminuer les revenus des ménages ainsi que leur consommation.

\appendix % Annexes, ARTICLE & BOOK

%\bibliography{•} % THE .BIB FILE HERE, WITHOUT THE EXTENSION
\printindex
\printglossaries

\end{document}
